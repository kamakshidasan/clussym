%\documentclass[journal]{vgtc}                % final (journal style)
\documentclass[review,journal]{vgtc}         % review (journal style)
%\documentclass[widereview]{vgtc}             % wide-spaced review
%\documentclass[preprint,journal]{vgtc}       % preprint (journal style)
%\documentclass[electronic,journal]{vgtc}     % electronic version, journal

%% Uncomment one of the lines above depending on where your paper is
%% in the conference process. ``review'' and ``widereview'' are for review
%% submission, ``preprint'' is for pre-publication, and the final version
%% doesn't use a specific qualifier. Further, ``electronic'' includes
%% hyperreferences for more convenient online viewing.

%% Please use one of the ``review'' options in combination with the
%% assigned online id (see below) ONLY if your paper uses a double blind
%% review process. Some conferences, like IEEE Vis and InfoVis, have NOT
%% in the past.

%% Please note that the use of figures other than the optional teaser is not permitted on the first page
%% of the journal version.  Figures should begin on the second page and be
%% in CMYK or Grey scale format, otherwise, colour shifting may occur
%% during the printing process.  Papers submitted with figures other than the optional teaser on the
%% first page will be refused.

%% These three lines bring in essential packages: ``mathptmx'' for Type 1
%% typefaces, ``graphicx'' for inclusion of EPS figures. and ``times''
%% for proper handling of the times font family.

\usepackage{mathptmx}
\usepackage[pdftex]{graphicx, color}
\usepackage{graphicx}
\usepackage{subfigure}
\usepackage{times}
\usepackage{amsmath}
\usepackage{float}
\usepackage{multirow}
%TODO
\usepackage{amsfonts}
%% We encourage the use of mathptmx for consistent usage of times font
%% throughout the proceedings. However, if you encounter conflicts
%% with other math-related packages, you may want to disable it.

%% If you are submitting a paper to a conference for review with a double
%% blind reviewing process, please replace the value ``0'' below with your
%% OnlineID. Otherwise, you may safely leave it at ``0''.
\onlineid{111}

%% declare the category of your paper, only shown in review mode
\vgtccategory{Algorithm/Technique}

%% allow for this line if you want the electronic option to work properly
\vgtcinsertpkg

%% In preprint mode you may define your own headline.
%\preprinttext{To appear in an IEEE VGTC sponsored conference.}

%% Paper title.

%% This is how authors are specified in the journal style

%% indicate IEEE Member or Student Member in form indicated below
%\author{Dilip Mathew Thomas, and Vijay Natarajan, \textit{Member, IEEE}}
%\authorfooter{
% insert punctuation at end of each item
%\item
% Dilip Mathew Thomas is with Department of Computer Science and Automation, Indian Institute of Science, Bangalore, India. E-mail: dilip@csa.iisc.ernet.in.
%\item
% Vijay Natarajan is with Department of Computer Science and Automation, Supercomputer Education Research Centre, Indian Institute of Science, Bangalore, India. E-mail: vijayn@csa.iisc.ernet.in.

%\item
% Martha Stewart is with Martha Stewart Enterprises at Microsoft
% Research, E-mail: martha.stewart@marthastewart.com.
%}

%other entries to be set up for journal
\shortauthortitle{Biv \MakeLowercase{\textit{et al.}}: Multiscale Symmetry Detection in Scalar Fields by Clustering Contours}
%\shortauthortitle{Biv \MakeLowercase{\textit{et al.}}: Global Illumination for Fun and Profit}
%\shortauthortitle{Firstauthor \MakeLowercase{\textit{et al.}}: Paper Title}

% ---------------------------------------------------------------------------
% Author guideline and sample document for EG publication using LaTeX2e input
% D.Fellner, v1.13, Jul 31, 2008
\begin{document}
\vspace{-1cm}
\begin{figure*}[t]
\onecolumn
\begin{center}
{\sffamily\huge{Multiscale Symmetry Detection in Scalar Fields\\
by Clustering Contours\\}}
\vspace{0.5cm}
{\sffamily\LARGE{Supplemental Figures\\}}
\end{center}
% for anonymous conference submission please enter your SUBMISSION ID
% instead of the author's name (and leave the affiliation blank) !!

%{\sffamily\huge{Detecting Symmetry in Scalar Fields Using Distance Weighted Extremum Graphs\\}}
%\vspace{0.5cm}
%{\sffamily\LARGE{Supplemental Figures\\}}
\vspace{0.5cm}
\centering
{\sffamily\large{1. Plot for Selecting Noise Parameter $\delta$}\\}
	\centering
		\includegraphics[scale=0.15]{figures/noiseplot.pdf}
\caption{The plot of the number of arcs in the contour tree with a specific value of volume 
	as the value of the volume increases. The curve shows a significant drop initially due to the arcs
corresponding to noise. A value immediately after this drop is chosen as the value of $\delta$. The above plot
corresponds to the Fuel dataset. Plots corresponding to the other datasets behave similarly.}
\end{figure*}
\begin{figure*}[t]
\centering
{\sffamily\large{2. Descriptor Space: Projection onto a 2D plane\\}}
\vspace{0.5cm}
	\centering
	\includegraphics[scale=0.3]{inkscape/1654mds.pdf}
\caption{Projection of 93 points in the descriptor space to 2D using multidimensional scaling
	for the dataset EMDB-1654. Observe that the clusters are well separated. For the two clusters within the
	dotted circles, their sizes are annotated below the cluster.}
\end{figure*}
\begin{figure*}
	\centering
	{\sffamily\large{3. Query Contour Driven Exploration\\}}
	\centering
	\subfigure[]
	{
		\includegraphics[scale=0.09]{figures/isabel12.png}
		\label{volisa12}
	}
	\subfigure[]
	{
		\includegraphics[scale=0.09]{figures/isabel24.png}
		\label{volisa24}
	}
	\subfigure[]
	{
		\includegraphics[scale=0.09]{figures/isabel36.png}
		\label{volisa36}
	}
	\subfigure[]
	{
		\includegraphics[scale=0.09]{figures/isabel48.png}
		\label{volisa48}
	}
\caption{\label{volisa}Low pressure regions extracted through query contour driven exploration of 
	the hurricane Isabel dataset. The yellow regions highlight the volumetric regions identified from 
	the time steps, 12, 24, 36, and 48, using the query contour from the first time 
	step shown in Figure~7(c). The extracted regions are displayed in the context
	of the rest of the volume.}
\end{figure*}
\begin{figure*}
\centering
\begin{minipage}{\textwidth}
	\centering
	\subfigure[]
	{
		\includegraphics[scale=0.18]{figures/pac18-full.png}
	}
	\subfigure[]
	{
		\includegraphics[scale=0.18]{figures/pac18ctrs.png}
	}
	\subfigure[]
	{
		\includegraphics[scale=0.18]{figures/pac19-sym.png}
	}
	\subfigure[]
	{
		\includegraphics[scale=0.18]{figures/pac19ctrs.png}
	}
	\subfigure[]
	{
		\includegraphics[scale=0.18]{figures/pac20-sym.png}
	}
	\subfigure[]
	{
		\includegraphics[scale=0.18]{figures/pac20ctrs.png}
	}
	\subfigure[]
	{
		\includegraphics[scale=0.18]{figures/pac21-sym.png}
	}
	\subfigure[]
	{
		\includegraphics[scale=0.18]{figures/pac21ctrs.png}
	}
	\subfigure[]
	{
		\includegraphics[scale=0.18]{figures/pac22-sym.png}
	}
	\subfigure[]
	{
		\includegraphics[scale=0.18]{figures/pac22ctrs.png}
	}
\end{minipage}
\caption{\label{pacdep}Query contour driven exploration using multiple contours on a weather simulation dataset of size $306\times285\times27$.
	(a)~Volume rendering of the pressure field from the first time step of the simulation shows two depressions in the Pacific ocean in red.
	(b)~Two query contours selected from the low pressure region of the first time step shown in gold and cyan.
	(c)-(j)~The result of the query on four subsequent time steps that are one hour apart shows the two low pressure regions
	detected in gold and cyan. The left and the right columns show the volumetric region extracted and the contour located, respectively.}
\end{figure*}
\begin{figure*}
\centering
{\sffamily\large{4. Multiscale Symmetry in Additional Datasets\\}}
\vspace{0.5cm}
\centering
\begin{minipage}{0.33\textwidth}
	\centering
	\subfigure
	{
		\includegraphics[scale=0.28]{figures/vortex-full.png}
	}
	\subfigure
	{
		\includegraphics[scale=0.28]{figures/vortex-sym1.png}
	}
	\subfigure
	{
		\includegraphics[scale=0.28]{figures/vortex-sym2.png}
	}
	\subfigure
	{
		\includegraphics[scale=0.28]{figures/vortex-sym3.png}
	}
\end{minipage}
\begin{minipage}{0.33\textwidth}
	\centering
	\subfigure
	{
		\includegraphics[scale=0.28]{figures/1179-full.png}
	}
	\subfigure
	{
		\includegraphics[scale=0.28]{figures/1179-sym3.png}
	}
	\subfigure
	{
		\includegraphics[scale=0.28]{figures/1179-sym2.png}
	}
	\subfigure
	{
		\includegraphics[scale=0.28]{figures/1179-sym1.png}
	}
\end{minipage}
\begin{minipage}{0.33\textwidth}
	\subfigure
	{
		\includegraphics[scale=0.32]{figures/neghip-full.png}
	}
	\subfigure
	{
		\includegraphics[scale=0.32]{figures/neghip-sym1.png}
	}
	\subfigure
	{
		\includegraphics[scale=0.32]{figures/neghip-sym2.png}
	}
\end{minipage}
\caption{\label{addres}Multiscale symmetry detected on (left column)~Vortex (middle column)~EMDB-1179 and (right column)~Neghip datasets.
	The topmost figure in each column shows a volume rendering of the dataset and different symmetric regions detected
are shown below it.}
\end{figure*}
\begin{figure*}
	\centering
	{\sffamily\large{5. Robustness\\}}
	\centering
	\subfigure[]
	{
		\includegraphics[scale=0.16]{figures/sph0.png}
		\label{sph0}
	}
	\subfigure[]
	{
		\includegraphics[scale=0.16]{figures/sph1.png}
		\label{sph1}
	}
	\subfigure[]
	{
		\includegraphics[scale=0.16]{figures/sph2.png}
		\label{sph2}
	}
	\subfigure[]
	{
		\includegraphics[scale=0.16]{figures/sph5.png}
		\label{sph5}
	}
	\subfigure[]
	{
		\includegraphics[scale=0.5]{figures/path4.png}
		\label{path4}
	}
	\caption{Robustness to noise. (a)~An isosurface of a synthetic dataset 
		which has the shape of a sphere. The number of critical points for this dataset
		is 16. (b)-(d)~Adding increasing levels of noise to the dataset deforms the
		isosurface. The number of critical points increase to 2673, 12690, and 80885.  
		(e)~For each datasets, the first ten non-zero eigen values of the Laplace-Beltrami spectra 
		are normalized by dividing with the first non-zero eigen value and plotted as a 1D curve. 
		The similarity of the curves shows that the similarity of the contour can be identified easily even in the presence
		of noise. }
\end{figure*}

\end{document}
