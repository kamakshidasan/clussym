%\documentclass[journal]{vgtc}                % final (journal style)
\documentclass[review,journal]{vgtc}         % review (journal style)
%\documentclass[widereview]{vgtc}             % wide-spaced review
%\documentclass[preprint,journal]{vgtc}       % preprint (journal style)
%\documentclass[electronic,journal]{vgtc}     % electronic version, journal

%% Uncomment one of the lines above depending on where your paper is
%% in the conference process. ``review'' and ``widereview'' are for review
%% submission, ``preprint'' is for pre-publication, and the final version
%% doesn't use a specific qualifier. Further, ``electronic'' includes
%% hyperreferences for more convenient online viewing.

%% Please use one of the ``review'' options in combination with the
%% assigned online id (see below) ONLY if your paper uses a double blind
%% review process. Some conferences, like IEEE Vis and InfoVis, have NOT
%% in the past.

%% Please note that the use of figures other than the optional teaser is not permitted on the first page
%% of the journal version.  Figures should begin on the second page and be
%% in CMYK or Grey scale format, otherwise, colour shifting may occur
%% during the printing process.  Papers submitted with figures other than the optional teaser on the
%% first page will be refused.

%% These three lines bring in essential packages: ``mathptmx'' for Type 1
%% typefaces, ``graphicx'' for inclusion of EPS figures. and ``times''
%% for proper handling of the times font family.

\usepackage{mathptmx}
\usepackage{graphicx}
\usepackage{times}
\usepackage{cite}
\usepackage{amsmath, amsfonts}
\usepackage{subfigure}
\usepackage{algorithm2e}

%% We encourage the use of mathptmx for consistent usage of times font
%% throughout the proceedings. However, if you encounter conflicts
%% with other math-related packages, you may want to disable it.

%% This turns references into clickable hyperlinks.
\usepackage[bookmarks,backref=true,linkcolor=black]{hyperref} %,colorlinks
\hypersetup{
  pdfauthor = {},
  pdftitle = {},
  pdfsubject = {},
  pdfkeywords = {},
  colorlinks=true,
  linkcolor= black,
  citecolor= black,
  pageanchor=true,
  urlcolor = black,
  plainpages = false,
  linktocpage
}

%% If you are submitting a paper to a conference for review with a double
%% blind reviewing process, please replace the value ``0'' below with your
%% OnlineID. Otherwise, you may safely leave it at ``0''.
\onlineid{0}

%% declare the category of your paper, only shown in review mode
\vgtccategory{Technique}

%% allow for this line if you want the electronic option to work properly
\vgtcinsertpkg

%% In preprint mode you may define your own headline.
%\preprinttext{To appear in an IEEE VGTC sponsored conference.}

%% Paper title.
\title{Symmetry Detection in Scalar Fields by Clustering Contours}


%% This is how authors are specified in the journal style
\author{Dilip Mathew Thomas and Vijay Natarajan, \textit{Member, IEEE}}
\authorfooter{
% insert punctuation at end of each item
\item
 Dilip Mathew Thomas is with Department of Computer Science and Automation, Indian Institute of Science, Bangalore, India. E-mail: dilip@csa.iisc.ernet.in.
\item
 Vijay Natarajan is with Department of Computer Science and Automation, and Supercomputer Education Research Centre, Indian Institute of Science, Bangalore, India. E-mail: vijayn@csa.iisc.ernet.in.
%\item
% Martha Stewart is with Martha Stewart Enterprises at Microsoft
% Research, E-mail: martha.stewart@marthastewart.com.
}

%% indicate IEEE Member or Student Member in form indicated below
%\author{Roy G. Biv, Ed Grimley, \textit{Member, IEEE}, and Martha Stewart}
%\authorfooter{
%% insert punctuation at end of each item
%\item
% Roy G. Biv is with Starbucks Research. E-mail: roy.g.biv@aol.com.
%\item
% Ed Grimley is with Grimley Widgets, Inc.. E-mail: ed.grimley@aol.com.
%\item
% Martha Stewart is with Martha Stewart Enterprises at Microsoft
% Research. E-mail: martha.stewart@marthastewart.com.
%}

%other entries to be set up for journal
\shortauthortitle{Biv \MakeLowercase{\textit{et al.}}: }
%\shortauthortitle{Firstauthor \MakeLowercase{\textit{et al.}}: Paper Title}

%% Abstract section.
\abstract{} % end of abstract

%% Keywords that describe your work. Will show as 'Index Terms' in journal
%% please capitalize first letter and insert punctuation after last keyword
\keywords{}

%% ACM Computing Classification System (CCS). 
%% See <http://www.acm.org/class/1998/> for details.
%% The ``\CCScat'' command takes four arguments.

%\CCScatlist{ % not used in journal version
%	\CCScat{Computer Graphics}{I.3.6}{Computer Graphics}{Methodology and Techniques}
%}
\teaser{
	\centering
	\subfigure[]
	{
		\includegraphics[scale=.33]{figures/1897-full.png}
	}
	\subfigure[]
	{
		\scalebox{.33}{\input{xfig/desspace.pdf_t}}
	}
	\subfigure[]
	{
		\includegraphics[scale=.33]{figures/1897-sym.png}
	}
	\caption{\label{teaser}Clustering based analysis detects symmetry at different scales in the electron microscopy 
	data of AMP-activated kinase (EMDB-1897). (a)~Volume rendering of the data shows different symmetric patterns
	that exhibit 3-fold rotational symmetry. (b)~Contours 
	are represented as points in a high dimensional descriptor space (shown in 2D for illustration) using
	a suitable shape descriptor. Contours belonging to symmetric regions are also symmetric and have similar descriptors.
	Hence, symmetric contours form a cluster in the descriptor space and can be easily identified.  
	Three such clusters detected are shown in golden, blue, and pink colours. (c)~Symmetric regions in the volume 
	corresponding to the symmetric contours are extracted and three different symmetric regions are highlighted 
	in golden, blue, and pink colours corresponding to the three clusters.}
}

%% Uncomment below to include a teaser figure.
%%  \teaser{
%%  \centering
%%  \includegraphics[width=16cm]{CypressView}
%%  \caption{In the Clouds: Vancouver from Cypress Mountain.}
%%  }

%% Uncomment below to disable the manuscript note
%\renewcommand{\manuscriptnotetxt}{}

%% Copyright space is enabled by default as required by guidelines.
%% It is disabled by the 'review' option or via the following command:
% \nocopyrightspace

%%%%%%%%%%%%%%%%%%%%%%%%%%%%%%%%%%%%%%%%%%%%%%%%%%%%%%%%%%%%%%%%
%%%%%%%%%%%%%%%%%%%%%% START OF THE PAPER %%%%%%%%%%%%%%%%%%%%%%
%%%%%%%%%%%%%%%%%%%%%%%%%%%%%%%%%%%%%%%%%%%%%%%%%%%%%%%%%%%%%%%%%

\begin{document}

%% The ``\maketitle'' command must be the first command after the
%% ``\begin{document}'' command. It prepares and prints the title block.

%% the only exception to this rule is the \firstsection command
%-------------------------------------------------------------------------
\firstsection{Introduction}
\maketitle
Many scientific experiments and simulations generate scalar field data and 
they often contain symmetric or repeating patterns. In many disciplines, 
symmetry plays an important role in studying the underlying scientific phenomena. 
For example, in crystallography, symmetry information is used to determine 
the structure of a crystal~\cite{som07}. In product design, symmetry is important 
to ensure functional efficiency and optimal manufacturing cost~\cite{booth02}. Symmetry is a 
useful cue in biology for determining growth and development of organs~\cite{stev06}. Since 
the study of symmetric features is of great interest in scientific data 
analysis, the problem of detecting symmetry in scalar fields has received 
considerable attention among researchers in the recent past~\cite{ThomN11,HongS08,kerbWKS11,ThomN13,MasoodTN13}.

Automatic detection of symmetry in scalar fields is a challenging problem and 
the quest for a widely applicable, efficient, and robust method for symmetry 
detection is ongoing. Though symmetry detection in scalar
fields is a relatively new area of research, the problem of detecting symmetry
in shapes has been well studied in the geometry processing community.
These studies have established that clustering based analyses result in superior
performance in robust identification of symmetry. Some of the methods
that use clustering for detecting symmetry in shapes have been extended to
scalar fields. These methods operate by determining symmetry
transformations through aggregation of local symmetry of sample points of the
domain. Scalar field datasets are typically represented using scalar values 
assigned to a discrete set of sample points that represent the domain under 
consideration. However, the domain and the scalar values are assumed to be continuous by
interpolating the values at the sample points. Therefore, the sample points in a scalar 
field capture the lowest level of information. In practice, scientists are more interested
in higher level features, extracted through methods like segmentation and isosurface extraction,
for studying the underlying physical phenomena. Hence, symmetry detection methods based on 
local information available at the sample points face considerable difficulty in representing 
and extracting meaningful symmetric regions. Moreover, these methods are computationally 
expensive since the number of sample points in scalar field datasets is typically orders of magnitude
higher than that in geometric shape datasets.

Though it is clear from methods proposed in the geometry processing community
that a clustering based analysis offers significant advantages in detecting symmetry,
we believe that unlike shapes, low-level information available at the sample points of the 
domain is not suited for symmetry detection in scalar fields. In this work, we propose 
a novel symmetry detection method based on the idea of clustering contours. Isosurfaces
are extensively used in studying scalar field datasets and contours capture information 
about a scalar field at a macroscale as opposed to sample points of the domain. Therefore, they
are more suitable for a clustering based analysis. It is easy to see that contours
belonging to regions with symmetric scalar field distribution are also symmetric. Using an 
appropriate shape descriptor, our method maps contours as points in the space of descriptors 
such that the distance between points in the descriptor space is a measure of similarity 
between the contours. As a result, points in the descriptor space 
corresponding to symmetric contours lie in close 
proximity to each other and form clusters in the descriptor space. The region of the domain
corresponding to each such contour can be extracted and these regions
are reported as symmetric as illustrated in Figure~\ref{teaser}. Note that the choice
of the shape descriptor is not fixed and depending on the noise characteristics and the notion
of similarity relevant to the application of interest, an appropriate descriptor may be used.

The main contributions of this paper are the following:
\begin{itemize}
\item A formulation of the problem of symmetry detection in scalar fields
as a clustering problem in a shape descriptor space. This model
provides a lot of flexibility in analysing similarity of scalar fields
as well as handling noise since it allows the shape representation and the 
descriptor space to be varied.
\item A novel representation of contours as points in a contour descriptor space.
Similarity between contours is naturally defined as the distance between points 
in this space. This is a generic representation of independent interest 
and we show its benefit in similarity analysis on scalar fields.
\item A robust algorithm to detect symmetric regions at multiple scales. Though geometry based 
symmetry detection methods are typically computationally costly, we design an efficient algorithm 
that employs elegant optimisations by incorporating topological information 
about the contours using the contour tree.
\item Applications to feature correspondence estimation and asymmetry visualization.
\end{itemize}
Symmetry information in scalar fields
has been used for transfer function design, exploration of isosurfaces, selection of cross-section
planes and view directions, linked selection and editing, query driven exploration,
and visualization of features through dual rendering~\cite{ThomN11,HongS08,ThomN13,MasoodTN13}.
We believe that as better techniques for robust detection of symmetry in scalar fields are developed,
many more applications will emerge.
\section{Related Work}
Existing symmetry detection methods in scalar fields can be broadly classified into two
categories, namely, geometry based methods and topology based methods. We briefly review
these methods in this section.
\subsection{Geometry based approaches}
Several methods have been proposed in the literature for detecting
symmetry in shapes and these methods are described in a survey paper by
Mitra~et~al.~\cite{MitPWC2012}. Some of these methods~\cite{Mitra06,KazhdanCDFR03,BokBWSS09} have 
been applied to scalar fields~\cite{MasoodTN13,HongS08,kerbWKS11}, however they struggle to address 
the challenges in extending geometric methods to scalar fields.
Scalar field datasets are significantly larger in size and hence symmetry detection is computationally costly.
Geometric methods typically consider geometric information derived from a small region around each sample
point of the domain for symmetry detection. Directly applying this approach to scalar field makes it difficult 
to capture important features in a scalar field and hence these methods perform poorly in extracting higher level
features of interest and handling noise in the data.
Moreover, the scalar values are assumed to be continuous over the domain by interpolating the values at 
the discrete set of sample points. Therefore, inspecting only local information at the sample points
introduces additional challenges due to discretization errors since the symmetric counterpart for a 
given point may be an interpolated point.

Hong and Shen~\cite{HongS08} propose a method to detect global reflective symmetry by identifying planes
of reflection that minimize the difference between the scalar value at a point and its
reflection. This method is computationally inefficient and cannot be easily extended to detect 
other types of symmetry. Kerber~et~al.~\cite{kerbWKS11} build a graph network of crease line features in a dataset
and detect symmetry by computing geometric transformations that match subgraphs within the crease line network.
Since only a small subset of features in scalar field datasets contain crease lines, 
this method is not very useful in practice. Masood~et~al.~\cite{MasoodTN13} detect
symmetry by identifying symmetry transformations as clusters in the space of all transformations.
The clusters are generated by aggregating local symmetry transformations of pairs of points in the 
domain. This method relies on local signatures of sample points for determining 
transformations and as a result several parameters need to be tweaked at various stages of 
the symmetry detection pipeline to limit the adverse effects of variations in the local signatures 
and discretization errors.  Moreover, the transformation space is often cluttered with a lot of 
spurious transformations and introduces artifacts in the extraction of symmetric regions from the point set. 
The above methods compute geometric transformations between candidate pairs for identifying
symmetry and are computationally costly. Moreover, these methods being purely driven by local geometric 
information, do not incorporate any notion of importance of a feature. Hence, they consider every 
candidate pair, including pairs corresponding to noise, to be equally important in their analysis. 
Our method, on the other hand, uses topological information derived from the contour tree to infer 
importance of a feature and this allows the design of a feature-aware algorithm for symmetry detection. 

Bruckner~et~al. propose an information theoretic approach for isosurface similarity 
detection~\cite{BrucknerM10,haidacher11}. This method uses mutual information of distance 
transforms to quantify the information common to two isosurfaces and build a similarity map 
between all pairs of isosurfaces. Clusters with high mutual information is used to detect 
similar isosurfaces within and across datasets. Although closely related to our method, 
by virtue of using distance transforms as the descriptor, 
this method identifies similar family of isosurfaces within a range of isovalues that 
form an onion-peel like layered arrangement whereas by identifying
similarity between contours and not isosurfaces, our method is more suited for identifying partial 
similarities and symmetric regions. While this method is limited to analysing similarity between
datasets only in pairs, the descriptor space can be used to analyse multiple datasets.
Similarity between different scalar fields has also been studied by measuring the extent of overlap
between contours~\cite{SchneiderWCHS08,schn13}. Changes in orientation affects the distance transform
descriptor and contour overlap thus affecting the similarity estimation whereas our method is not tied 
to a particular choice of descriptor. Based on the requirements of the application under 
consideration, our method  can be adapted to be sensitive or insensitive to orientation. 
\subsection{Topology based methods}
Thomas and Natarajan propose topology based methods for symmetry detection and these methods 
are computationally efficient because they operate on graph representations of the scalar field
like the contour tree~\cite{ThomN11} and the extremum graph~\cite{ThomN13}. The contour tree based method 
assumes that the subtrees of the contour tree corresponding to symmetric regions are structurally
similar. They detect symmetry by evaluating structural similarity between the subtrees using a 
similarity score. This method can detect symmetry at multiple scales but cannot handle
noise in scalar fields that destroy the repeating structure of the subtrees. The extremum 
graph based method selects a set of extrema called seed set and 
estimates distances robustly through a graph traversal procedure. A carefully chosen distance 
threshold is used to disconnect the graph and classify the seeds into different groups called 
super-seeds. A region growing procedure is then used to identify the symmetric region 
corresponding to each super-seed. This procedure makes a strong assumption that the symmetric regions
can be identified purely from the proximity relationship between the seeds. Hence,
it relies heavily on a meaningful selection of seed set which involves significant effort 
and understanding about the symmetry of the domain. Additionally, this method requires several thresholds
to be set.
\begin{figure}[b]
\centering
{
	\scalebox{0.3}{\input {xfig/ctrmap.pdf_t}}
	\caption{\label{ctmap}Mapping of contours as points in the descriptor space. Symmetric contours
		shown in blue colour are mapped to the same blue point in the descriptor space. Six approximately
		symmetric contours shown in golden colour are mapped to six points which lie in close
		proximity to each other in the descriptor space.}
}
\end{figure}

The above two methods being topological in nature do not ensure that the regions detected by
them are indeed geometrically symmetric while our method being geometric in nature, 
ensures that the regions detected are symmetric. Moreover, they compare candidate regions 
pairwise and rely on a similarity threshold to classify them into symmetric groups. Determining
the similarity threshold is a challenge when using datasets with varying characteristics.
Clustering based analysis avoids the need for pairwise comparisons. Instead, the symmetric regions
are directly obtained as clusters in the descriptor space. 
Similarity between scalar fields have been
studied in the context of shape matching applications by using graph matching methods on 
discrete approximations of the contour tree~\cite{ZhangBKDNB06,HilagaSKK01}.
The contour tree provides metadata information about the contours
and allows integration of topological information in geometric processing. Thus, by utilizing 
geometric information of contours and the power of contour tree as a topological abstraction of 
contours, our method offers significant advantages over existing symmetry detection methods. 
\begin{figure*}[t]
	\centering
	\scalebox{0.2}{\input {xfig/pipeline.pdf_t}}
	\caption{\label{pipeline} Symmetry detection pipeline. Contours are extracted
		from the scalar field dataset and a descriptor is generated for each contour.
		A correspondence score is estimated between pairs of contours based on the
		distance between the points in the descriptor space. Next, the set of symmetric
		contours are identified through clustering and the region of the domain corresponding
	to each symmetric contour is extracted and reported.}
\end{figure*}
\section{Definitions}
Consider a \emph{scalar field}, $f : \mathbb{M}  \rightarrow \mathbb{R}$, defined on a 
simply connected domain $\mathbb{M}$. The preimage of $f$ for a given value $u \in \mathbb{R}$, 
$f^{-1}(u)$, is called a \emph{level set} of $f$. A level set may have multiple components
and each component is called a \emph{contour}. Consider a sweep of the level sets in the order
of increasing function values. A contour is created at a \emph{minimum}, may merge
with another contour or split into different contours at a \emph{saddle}, and is destroyed
at a \emph{maximum}. The \emph{contour tree} is a topological 
data structure that captures these changes in the connectivity of the level sets. 
Let $R$ be an equivalence relation defined on points in $\mathbb{M}$: $xRy$ for $x,y 
\in \mathbb{M}$  if $x,y$ belong to the same contour. The contour tree is the quotient space 
induced by this relation. 

Subdomains $\mathbb{M}_1, \mathbb{M}_2 \subseteq \mathbb{M}$ are said to be \emph{symmetric} 
if there is a transformation $T$ such that ${\mathbb{M}_2=T(\mathbb{M}_1)}$ and 
${f(x)=f(T(x))}$ for all $x \in \mathbb{M}_1$. If $c_1$ and $c_2$ are contours of the same level
set that belong to $\mathbb{M}_1$ and $\mathbb{M}_2$ respectively, it is easy to see that if 
$\mathbb{M}_1$ and $\mathbb{M}_2$ are symmetric then $c_1$ and $c_2$ are also symmetric. 
We make use of this property and detect symmetric subdomains by identifying symmetry of the 
contours belonging to the subdomains. The above definition of symmetry requires computation of the symmetry 
transformation $T$ which is a costly operation. Let $\mathbb{C}$ be the set of all contours. 
Consider a function $g : \mathbb{C} \rightarrow \mathbb{R}^n$ such that $g(c) = g(T(c))$
where $T$ is any transformation. In other words, $g$ is a function that maps each
contour to a point in a high dimensional space, called a \emph{descriptor}, 
such that a contour and its symmetric copies are mapped to the same point, see Figure~\ref{ctmap}.
For illustration, the descriptor space is shown in 2D but the actual dimension of the space
depends on the choice of the descriptor. We call this high dimensional space the
\emph{descriptor space}. Thus, if $c_1$ and $c_2$ are perfectly symmetric,
the distance between them in the descriptor space is zero, i.e., 
$\lVert g(c_1)-g(c_2) \rVert = 0$, where $\lVert \cdot \rVert$ is a norm in
the descriptor space. Shape descriptors have been extensively used in the
geometry processing community for shape matching and there is a
vast collection of research papers in this 
area~\cite{lian2013,van2011,tangelder2008survey,qin2008content}. Note that
different norms may have to be used for different descriptors. 

In practice, scalar field datasets do not exhibit perfect symmetry and therefore
it is important to detect symmetry in an approximate sense. Ideally, deviation
from perfect symmetry should be measured in the space of shapes but in practice,
it is more convenient to measure deviations in the descriptor space. If contours $c_1$ and 
$c_2$ deviate from perfect symmetry, then $c_1$ and $c_2$ will not be mapped
to the same point in the descriptor space and the distance between
them in the descriptor space, $\lVert g(c_1)-g(c_2) \rVert$, will be indicative
of the deviation from perfect symmetry. We say that $c_1$ and $c_2$ are
\emph{$\epsilon$-symmetric} if the distance between $c_1$ and $c_2$ in the descriptor 
space is less than $\epsilon$, i.e., $\lVert g(c_1)-g(c_2) \rVert < \epsilon$.
A shape descriptor may map contours $c_1$ and $c_2$ to the same point
even when they have totally different shapes. A good shape descriptor should 
discriminate between different shapes well and reduce the occurrence of such incorrect
mappings. 
\section{Symmetry Detection via Contour Clustering}
Methods based on clustering~\cite{Mitra06,Lip10,MitraGP07,Xu12,RavBBK10,Xu09}
have shown superior performance in detecting
symmetry in shapes. However, directly extending these methods to scalar fields
is not easy. In this section, we describe a novel symmetry detection method
based on the idea of clustering contours of the scalar field.
\subsection{Overview}
The main steps of our algorithm is shown in Figure~\ref{pipeline}.
Given a scalar field as input, we generate a set of contours. For each contour
thus generated, a descriptor is computed. The descriptor for each contour
can be considered to be a point in a high dimensional 
descriptor space. The descriptor space is a transformation invariant space,
i.e., it reverses the effect of geometric transformation on contours. Thus, perfectly
symmetric contours are mapped to the same point in the descriptor space.
Imperfections in symmetry results in imperfections in the mapping.
Since contours with similar shape have similar descriptors,
the points in the descriptor space corresponding to approximately symmetric 
contours will lie in close proximity to each other. Therefore, symmetric contours
can be detected by identifying clusters in the descriptor space. The volumetric
region represented by each contours in a cluster is then reported as
symmetric regions.
\begin{figure}[t]
	\centering
	\subfigure[]
	{
		\scalebox{0.2}{\input {xfig/sam-ct.pdf_t}}
		\label{sam-ct}
	}
	\subfigure[]
	{
		\includegraphics[scale=0.1]{figures/ctrs-before-merge.png}
		\label{ct-bef-merge}
	}
	\hspace{0.2cm}
	\subfigure[]
	{
		\includegraphics[scale=0.1]{figures/ctrs-after-merge.png}
		\label{ct-aft-merge}
	}
	\caption{(a)~A coarse sampling of contours may fail to generate 
		contours from the region represented by the arc $ac$ of the contour tree
		and the region will not be considered for symmetry analysis.
		On the other hand, a fine sampling of contour may lead to the generation
		of multiple contours from the region represented by the arc $cd$ and
	result in redundant computations. The six contours in~(b) merge in pairs to form
the three contours in~(c). Since the shape of the individual contours before and after the merge is 
very different, two clusters are formed in the descriptor space, one corresponding to the
six contours before merge and the second corresponding to the three contours after merge.}
\end{figure}
\subsection{Contour Generation}\label{congen}
Isosurface extraction is a well understood topic since it is one of the basic
operations performed while studying scalar field data. Our algorithm assumes that
each region of interest in the domain is represented by a contour belonging to it.
Hence, it is important to use a sampling strategy that generates a contour from each
region of interest. The obvious method for generating contours is 
to sample isovalues uniformly from the range of the function values and extract contours 
corresponding to these isovalues. However, uniform sampling has several drawbacks as shown 
in Figure~\ref{sam-ct}. A coarse uniform sampling may result in no contours getting 
generated from a region and thus fail to detect it as a symmetric region. On the other hand, 
a fine sampling will lead to multiple contours getting generated from the same region 
leading to redundant computations. Ideally, each symmetric region should require only a single 
representative contour belonging to the region to be processed.
The contour tree is a powerful tool that encapsulates information about the evolution of contours 
and we leverage information obtained from the contour tree for optimal generation of contours. 
Each arc of the contour tree represents a family of contours that are nested one inside the other.
Each contour in this family is typically a scaled version of the contour nested inside it. 
Hence, to capture the geometry of the region of the domain corresponding to an arc of the contour tree, 
we select only one contour from each arc of the contour tree.

Selecting a representative contour from each arc of the contour tree ensures that no regions
are missed in the subsequent symmetry analysis. However, for noisy scalar fields, a large
number of arcs of the contour tree may correspond to noise. Selecting a contour from each
arc of the contour tree will result in significant amount of computational time spent in
processing these noisy contours and will lead to poor performance. To overcome this problem,
we generate a contour from from an arc of the contour tree only if the arc is is deemed represent
a feature and not noise. The definition of noise is subjective and depends on the application.
In this work, we consider an arc to be noise if the volume of the 
largest contour associated with the arc is below a user defined threshold $\delta$. The volume of a
contour is approximated as the number of vertices of the domain enclosed by the contour. 
Since each arc of the contour tree can be associated with the set of vertices of the domain 
that comprise the subvolume corresponding to the arc, this estimation of the volume can be done 
efficiently~\cite{CarrSP10}. In the absence of the metadata information provided by the contour tree 
all contours would have had to be treated as equally important and it would neither have been possible to ensure
that that each region had a representative contour nor suppress generation of multiple contours from the 
same region or contours from noisy regions.
\subsection{Contour Representation in Descriptor Space}
Once representative contours are generated from each region of interest, the next step
is to generate a descriptor for each contour. The correspondence score between a pair of
contours is estimated using the distance between their descriptors in the descriptor space.
Designing shape descriptors for matching and retrieving similar shapes is a well studied 
area in the geometry processing community. 
The notion of similarity is subjective and varies from application to application.
One of the major advantages of our method is that it is not tied to a particular choice
of the shape descriptor. Instead, our method offers flexibility in choosing 
the right descriptor for an application.  Hence, our method can be used as a generic framework for detecting 
similar regions in a scalar field. For example, if an application is interested in 
identifying similarity only with respect to rotation, a rotation invariant descriptor may be used. 
The only prerequisite on the descriptor is that it should be discriminative, i.e., contours which are 
similar should be mapped to points that are nearby in the descriptor space while contours that differ 
from each other should be mapped to far away points. Therefore, it is important to use shape descriptors 
with high precision and recall ratios~\cite{lian2013} for applications that cannot tolerate false 
positives during shape retrieval.
\subsection{Contour Clustering}
Shape matching is typically performed on pairs of shapes by comparing the corresponding descriptors. 
Symmetry detection involves identification of a collection of similar. Therefore, 
a clustering based approach is more appropriate for this purpose, 
as opposed to an approach based on pairwise comparisons.
As described earlier, the descriptor space is a space in which the contours are represented as 
points in a transformation invariant manner and hence symmetric contours are mapped to points 
that lie in close proximity to each other. As a result, symmetric contours can be easily 
identified as clusters in the descriptor space. Strictly speaking, such a mapping of contours to points in the
descriptor space is not a prerequisite for clustering. A symmetry correspondence
graph can be constructed from a set of contours by representing each contour
as a node in the graph and inserting an edge between two nodes if the corresponding
contours are symmetric. It is easy to see that the contours that are symmetric form
a clique under this representation and these cliques can be identified to detect
symmetric contours~\cite{Lip10}. Thus, given a procedure that assigns a similarity 
correspondence score between a pair of contours, the symmetry correspondence graph
can be constructed. Further processing is performed solely on this graph and is independent
of the actual definition of similarity which may be application dependent. This allows
considerable freedom in choosing a similarity measure that is relevant to an application.
For symmetry detection, distance between points in the descriptor space is 
used to assign the correspondence score between pairs of contours. 

Given a set of contours identified to be symmetric, the arc in the contour tree 
corresponding to each contour can be determined. The region of the domain
enclosed by the largest contour of the arc can then be extracted and reported
as a symmetric region. Although this works well in practice, note that the contour
itself may have evolved from the merging of contours nested within it, see Figure~\ref{ct-bef-merge}
and~\ref{ct-aft-merge}.
An application with stricter requirements on symmetry detection may need to
also incorporate the symmetry of these nested contours in the algorithm. 
This presents a challenge in directly using clusters in the descriptor space 
for detecting symmetric region because the descriptor space is not continuous with 
respect to the evolution of the shape of a contour during a level set sweep. 
Recall that a cluster in the descriptor space corresponds to a
set of contours with the same shape. As shown in the illustration in 
Figure~\ref{ct-bef-merge} and~\ref{ct-aft-merge}, the shape of each individual nested contour
before merge is totally different different from the shape of each contour after the merge.
Therefore the points corresponding to the contours before and after the merge may not be
part of the same cluster. 
\begin{figure}[h]
\centering
\scalebox{.15}{\input{xfig/bip.pdf_t}}
\caption{\label{bip}Maximum weight matching is used to determine the contribution of children
	contours towards the correspondence score between the parent contours $p$ and $q$. The
	first and second partitions of the bipartite graph represents the children contours of $p$
	and $q$ respectively. Each edge is weighted with the correspondence score between the
	contours represented by the end points of the edge.}
\end{figure}

To address this issue, we incorporate the correspondence score between the children contours 
(contours before merging) while calculating the correspondence score between a given pair
of parent contours (contours after merging). Let $p$ and $q$ be two parent contours
with children contours ${c_1}^p,\dots,{c_n}^p$ and ${c_1}^q,\dots,{c_m}^q$ respectively. 
Assume that the correspondence score between any two
children contours ${c_i}^p$ and ${c_j}^q$ is known - for contours with children,
the procedure below can be applied bottom up to determine their correspondence
while for contours that do not have children, the correspondence score can be directly 
determined from the descriptor space. To calculate the correspondence score
between $p$ and $q$, first the contribution from the children contours 
is determined. We construct a bipartite graph where nodes in the first
and second partitions are ${c_1}^p,\dots,{c_n}^p$ and ${c_1}^q,\dots,{c_m}^q$ 
respectively, see Figure~\ref{bip}. An edge between ${c_i}^p$ and ${c_j}^q$ is weighted
with the correspondence score between ${c_i}^p$ and ${c_j}^q$. The maximum weight
matching can then be computed and gives the correspondence score between the children
contours. The correspondence score between the contours $p$ and $q$ obtained
directly from the descriptor space is added to the maximum weight matching score
with appropriate normalization and used as the cumulative
correspondence score between $p$ and $q$. See Section~\ref{clust} for 
the details of the normalization scheme we use.
\section{Implementation}
In the previous section, we described the generic details of our symmetry
detection algorithm. In this section, we elaborate on the details of our 
implementation of the algorithm.
\subsection{Isovalue Selection}
In Section~\ref{congen}, we explained that the contour tree facilitates smart
sampling of contours by selecting a contour from each arc of the contour tree
that is not deemed as noise. Each arc of the contour tree encodes the range
of values of the scalar field restricted to the subdomain represented by the arc.
The question that remains is which function value in this range should be used
to generate the representative contour for the arc. Rather than selecting any 
value, it is beneficial to employ a stability criterion and select the
maximally stable contour from each arc~\cite{MatasCUP04}. This ensures consistency
in the shape of the contours extracted from different regions that are symmetric.
The important observation here is that the contours within the same arc
may undergo changes to its shape just before (or soon after) it merges 
with another contour at a saddle. Similarly, the shape of the contour in the
immediate neighbourhood of an extremum is typically very small and undergoes
changes as it evolves into a bigger contour. 
%\begin{algorithm}[b]
%	\SetAlgoLined
%	add arcs into a queue Q in the decreasing order\\
%	$\;\;$of the higher valued end-point\\
%	$lm$ = Q.pop() \tcp*[f]{assume $f(l) > f(m)$}\\
%	\Repeat{Q.notempty()}
%	{
%		$w = f(l) - \alpha$\\
%		\While{Q.notempty() and $f(m)+\alpha <= w <= f(l)-\alpha$}
%		{
%			extract contour from arc $lm$ at isovalue $w$\\
%			$lm$ = Q.pop()\\
%		}
%	}
%	\caption{\label{samalgo}{
%			Isovalue selection algorithm. Assume $f(l) > f(m)$ for each arc $lm$ incident
%			on a join saddle. Each arc $lm$ is added into a queue in the decreasing order of 
%			$f(l)$. When an arc $lm$ is popped out of the queue, if the isovalue $w = f(l)-\alpha$ 
%			lies in between $f(l)$ and $f(m)$ but is not too close to either $f(l)$ or $f(m)$, 
%			then a contour is extracted at isovalue $w$ for all arcs that satisfy the criterion. 
%			When an arc does not satisfy the criterion, the process is repeated with a new isovalue $w$.}
%}
%\end{algorithm}

For the purpose of computational efficiency, we adopt a simple strategy
based on the above observation that contours in the immediate vicinity
of a saddle or an extremum are not stable. A parameter $\alpha$ is chosen
such that for each arc $lm$ the isovalue $w$ satisfies the condition
$f(m)+\alpha <= w <= f(l)-\alpha$, assuming $f(l) > f(m)$. In other words,
the isovalue $w$ is chosen such that it is not too close to either
$f(l)$ or $f(m)$. It is important
to note that the exact value of $\alpha$ is not critical because within an arc,
only those contours generated from isovalues that are very close to an extremum or
saddle may undergo significant changes in shape. Therefore, the exact value of $\alpha$
is not important as long as it is not set to a value too close to zero.
In our experiments we set $\alpha$ to 1\% of the range of the function values. 
%Consider a level set sweep in the increasing order of function values.
%The saddle at which a contour, created at a minimum, merge
%into another contour is called a join saddle. The algorithm is described
%for join saddles and arcs incident on join saddles. A split saddle is defined
%analogously by reversing the direction of the sweep and a similar algorithm
%is used to generate contours corresponding to arcs incident on the split saddles.
%The algorithm considers each join saddle $s_i$ in the decreasing order of function
%values. It then generates an isovalue $w = f(s_i)-\alpha$, where $\alpha$ is a
%parameter chosen to ensure that $w$ is not too close to $f(s_i)$. It is important
%to note that the exact value of $\alpha$ is not critical because within an arc,
%only those contours generated from isovalues that are very close to an extremum or
%saddle may undergo significant changes in shape. In our experiments we set $\alpha$
%to 1\% of the range of the function values. Next, each non-noisy arc $lm$ 
%incident on a join saddle is considered. If $w$ lies between $f(l)$ and $f(m)$ but not 
%too close to either $f(l)$ or $f(m)$, i.e., $f(m)+\alpha <= w <= f(l)-\alpha$, 
%then a contour is generated at the isovalue $w$ from the arc $lm$.
%
\subsection{Shape Descriptor}
As described, the choice of shape descriptor depends on the kind of similarity
analysis required by an application. For symmetry detection, we use the 
Laplace-Beltrami spectra as the shape descriptor. The descriptor has been used for 
shape matching and retrieval and is robust in the 
presence of noise, deformations in shape, and differences in the underlying 
triangulation~\cite{reuter2006laplace,niethammer2007global,reuter2009laplace}. 
Although it is possible that two different shapes may have the same spectra,
it is very rare in practice~\cite{reuter2006laplace} and is quite discriminative
with high precision and recall ratios~\cite{lian2013}. We use the popular cotangent
weighted scheme for computing the Laplace-Beltrami spectra~\cite{PinkallP93}.
Computation of the Laplace-Beltrami spectra on large meshes is costly and therefore
we simplify the contour meshes so that the number of vertices in the mesh is around 2000.
\subsection{Clustering}\label{clust}
%All text with the exception of the abstract must be in a two-column format.
%The total allowable width of the text area -- including header and footer
%lines -- is 161\,mm (6.34 inch) wide by 231\,mm (9.10 inch) high.
%
%Columns are to be 76\,mm (3.0 inch) wide, with a 8\,mm (0.315 inch) space 
%between them.
%
%The space between the header line and the first line of the text body and
%between the last line of the text body and the footer line is 5\,mm
%(0.196 inch) each.
%
%%-------------------------------------------------------------------------
%\subsection{Type-style and fonts}
%
%Wherever Times is specified, Times Roman may also be used. If
%neither is available on your word processor, please use the font
%closest in appearance to Times that you have access to. Only
%Type-1 fonts will be accepted.
%
%MAIN TITLE. The title should be in Times 17-point, boldface type and
%centered. Capitalize the first letter of nouns, pronouns, verbs, adjectives,
%and adverbs; do not capitalize articles, coordinate conjunctions, or
%prepositions (unless the title begins with such a word). Leave two blank
%lines after the title.
%
%AUTHOR NAME(s) and AFFILIATION(s) are to be centered beneath the title and
%printed in Times 9-point, non-boldface type. This information is to be
%followed by two blank lines.
%
%The ABSTRACT ist to be in a one-column format. The MAIN TEXT is to be in a
%two-column format.
%
%MAIN TEXT. Type main text in 9-point Times, single-spaced. Do \emph{not} use
%double-spacing. All paragraphs should be indented 1 em (the length of the
%dash in the actual font). Make sure your text is fully justified -- that is,
%flush left and flush right. Please do not place any additional blank lines
%between paragraphs. Figure and table captions should be 9-point Times
%boldface type as in Figure~\ref{fig:firstExample}.
%
%\noindent Long captions should be set as in Figure~\ref{fig:ex1} or
%Fig.~\ref{fig:ex3}.
%
%\begin{figure}[htb]
%   % an empty figure just consisting of the caption lines
%   \caption{\label{fig:ex1}
%     'Empty' figure only to serve as an example of long caption requiring 
%     more than one line. It is not typed centered but aligned on both sides.}
%\end{figure}
%
%\noindent
%Figures which need the full textwidth can be typeset as Fig.~\ref{fig:ex3}.
%
%\noindent Callouts should be 9-point Times, non-boldface type. Initially
%capitalize only the first word of section titles and first-, second-, and
%third-order headings.
%
%FIRST-ORDER HEADINGS. (For example, \textbf{1. Introduction}) should be Times
%9-point boldface, initially capitalized, flush left, with one blank line
%before, and one blank line after.
%
%SECOND-ORDER HEADINGS. (For example, \textbf{2.1. Language}) should be Times
%9-point boldface, initially capitalized, flush left, with one blank line
%before, and one after. If you require a third-order heading (we discourage
%it), use 9-point Times, boldface, initially capitalized, flush left, preceded
%by one blank line, followed by a period and your text on the same line.
%
%The headline \emph{(authors / title)} must be shortened if it uses the full 
%two column width of the main text.
%There must be enough space for the page numbers. Please use ``et al.'' if 
%there are more than three authors and specify a shortened version for your title.
%%-------------------------------------------------------------------------
%\subsection{Footnotes}
%
%Please do \emph{not} use footnotes at all!
%
%
%%-------------------------------------------------------------------------
%\subsection{References}
%
%List all bibliographical references in 9-point Times, single-spaced, at the
%end of your paper in alphabetical order. When referenced in the text, enclose
%the citation index in square brackets, for example~\cite{Lous90}. Where
%appropriate, include the name(s) of editors of referenced books.
%
%For your references please use the following algorithm:
%\begin{itemize} 
%\item \textbf{one} author: first 3 chars plus year -- 
%      e.g.\ \cite{Lous90}
%\item \textbf{two}, \textbf{three} or \textbf{four} authors: first char
%      of each family name plus year --  e.g.\ \cite{Fellner-Helmberg93} 
%      or \cite{Kobbelt97-USHDR} or \cite{Lafortune97-NARF}
%\item \textbf{more than 4} authors: first char of family name from 
%      first 3 authors followed by a '*' followed by the year -- 
%%      e.g.\ \cite{Buhmann:1998:DCQ} or \cite{FolDamFeiHug.etal93} 
%\end{itemize}
%
%For BibTeX users a style file \ \texttt{eg-alpha.bst} \ is available which
%uses the above algorithm.
%
%%-------------------------------------------------------------------------
%\subsection{Illustrations, graphs, and photographs}
%
%All graphics should be centered.
%
%%%%
%%%% Figure 1
%%%%
%\begin{figure}[htb]
%  \centering
%  % the following command controls the width of the embedded PS file
%  % (relative to the width of the current column)
%  \includegraphics[width=.8\linewidth]{sampleFig}
%  % replacing the above command with the one below will explicitly set
%  % the bounding box of the PS figure to the rectangle (xl,yl),(xh,yh).
%  % It will also prevent LaTeX from reading the PS file to determine
%  % the bounding box (i.e., it will speed up the compilation process)
%  % \includegraphics[width=.95\linewidth, bb=39 696 126 756]{sampleFig}
%  %
%  \parbox[t]{.9\columnwidth}{\relax
%           For all figures please keep in mind that you \textbf{must not}
%           use images with transparent background! 
%           }
%  %
%  \caption{\label{fig:firstExample}
%           Here is a sample figure.}
%\end{figure}
%
%If your paper includes images, it is very important that they are of
%sufficient resolution to be faithfully reproduced.
%
%To determine the optimum size (width and height) of an image, measure
%the image's size as it appears in your document (in millimeters), and
%then multiply those two values by 12. The resulting values are the
%optimum $x$ and $y$ resolution, in pixels, of the image. Image quality
%will suffer if these guidelines are not followed.
%
%Example 1: 
%%
%An image measures 50\,mm by 75\,mm when placed in a document. This
%image should have a resolution of no less than 600 pixels by 900
%pixels in order to be reproduced faithfully.
%
%Example 2: 
%%
%Capturing a screenshot of your entire $1024 \times 768$ pixel display
%monitor may be useful in illustrating a concept from your research. In
%order to be reproduced faithfully, that $1024 \times 768$ image should
%be no larger than 85 mm by 64 mm (approximately) when placed in your
%document.
%
%
%%-------------------------------------------------------------------------
%\subsection{Color}
%
%\textbf{Please observe:} as of 2003 publications in the proceedings of the
%Eurographics Conference can use color images throughout the paper. No
%separate color tables are necessary.
%
%However, workshop proceedings might have different agreements! 
%Fig.~\ref{fig:ex3} is an example for creating color plates.
%
%%------------------------------------------------------------------------
%\subsection{Embedding of Hyperlinks / Typesetting of URLs}
%
%Due to the use of the package \texttt{hyperref} the original behavior
%of the command $\backslash$\texttt{url} from the package \texttt{url}
%is not available. To circumvent this problem we either recommend to
%use the command $\backslash$\texttt{httpAddr} from the 
%included package \texttt{egweblnk} (see below) or to replace the
%command $\backslash$\texttt{url} by the command $\backslash$\texttt{webLink} 
%-- e.g. in cases where $\backslash$\texttt{url} has been used
%widely in BibTeX-References. In the latter case we suggest to run
%BibTeX as usual and then replace all occurences of $\backslash$\texttt{url}  by
%$\backslash$\texttt{webLink}
%
%\noindent
%The provided commands for hyperlinks, in a nutshell, are:
%
%\begin{description} \itemsep 1ex
%\item [\webLinkFont $\backslash$httpAddr \{URL without leading 'http:'\}]
%      \mbox{}\\
%      e.g. \  \httpAddr{//diglib.eg.org/EG/DL/WS}
%
%\item[\webLinkFont $\backslash$ftpAddr \{URL without leading 'ftp:'\}]
%      \mbox{}\\
%      e.g. \  \ftpAddr{//www.eg.org/EG/DL/ftpupload}
%
%\item[\webLinkFont $\backslash$URL \{url\}]
%      \mbox{}\\
%      e.g. \  \URL{http://www.eg.org/EG/DL/WS}
%
%\item[\webLinkFont $\backslash$MailTo \{Email addr\}]
%      \mbox{}\\
%      e.g. \  \MailTo{publishing@eg.org}
%
%\item[\webLinkFont $\backslash$MailToNA \{emailName\}\{@emailSiteAddress\}]
%      \mbox{}\\
%      e.g. \  \MailToNA{publishing}{@eg.org}
%
%\item[\webLinkFont $\backslash$webLink\{URL without hyperlink creation\}]
%      \mbox{}\\
%      e.g. \  \webLink{http://www.eg.org/some_arbitrary_long/but_useless/URL}
%
%\end{description}
%
%
%%------------------------------------------------------------------------
%\subsection{PDF Generation}
%
%Your final paper should be delivered as a PDF document with all typefaces
%embedded. \LaTeX{} users should use \texttt{dvips} and \texttt{ps2pdf} to
%create this PDF document. Adobe Acrobat Distiller may be used in place of
%\texttt{ps2pdf}.
%
%Adobe PDFWriter is \emph{not} acceptable for use. Documents created with
%PDFWriter will be returned to the author for revision. \texttt{pdftex} and
%\texttt{pdflatex} (and its variants) can be used only if the author can
%make certain that all typefaces are embedded and images are not downsampled
%or subsampled during the PDF creation process.
%
%Users with no access to these PDF creation tools should make available a
%PostScript file and we will make a PDF document from it.
%
%
%The PDF file \emph{must not} be change protected.
%
%%------------------------------------------------------------------------
%\subsubsection*{Configuration Notes: dvips / ps2pdf / etc.}
%
%\noindent
%\texttt{dvips} should be invoked with the \texttt{-Ppdf} and \texttt{-G0}
%flags in order to use Type 1 PostScript typefaces:
%
%\begin{verbatim}
%    dvips -t a4 -Ppdf -G0 -o my.ps my.dvi
%\end{verbatim}
%
%
%\noindent
%If you are using version 7.x of GhostScript, please use the following method of invoking \texttt{ps2pdf}, in
%order to embed all typefaces and ensure that images are not downsampled or subsampled in the PDF
%creation process:
%
%\begin{verbatim}
%  ps2pdf -dMaxSubsetPct=100 \
%         -dCompatibilityLevel=1.3 \
%         -dSubsetFonts=true \
%         -dEmbedAllFonts=true \
%         -dAutoFilterColorImages=false \
%         -dAutoFilterGrayImages=false \
%         -dColorImageFilter=/FlateEncode \
%         -dGrayImageFilter=/FlateEncode \
%         -dMonoImageFilter=/FlateEncode \
%         mypaper.ps mypaper.pdf
%\end{verbatim}
%
%
%If you are using version 8.x of GhostScript, please use this method in place of the example above:
%\begin{verbatim}
%  ps2pdf -dPDFSETTINGS=/prepress \
%         -dCompatibilityLevel=1.3 \
%         -dAutoFilterColorImages=false \
%         -dAutoFilterGrayImages=false \
%         -dColorImageFilter=/FlateEncode \
%         -dGrayImageFilter=/FlateEncode \
%         -dMonoImageFilter=/FlateEncode \
%         -dDownsampleColorImages=false \
%         -dDownsampleGrayImages=false \
%         mypaper.ps mypaper.pdf
%\end{verbatim}
%
%%------------------------------------------------------------------------
%\subsubsection*{Configuration Notes: pdftex / pdflatex / etc.}
%
%\noindent
%Configuration of these tools to embed all typefaces can be accomplished by editing the \texttt{updmap.cfg} file
%to enable inclusion of the standard (or base) 14 typefaces.
%
%Linux users can run the \texttt{updmap} script to do this:
%\begin{verbatim}
%updmap --setoption pdftexDownloadBase14 true
%\end{verbatim}
%
%Windows users should edit the \texttt{updmap.cfg} files found in their TeX installation directories (one or both
%of the following may be present):
%\begin{verbatim}
%  INSTALLDIR\texmf\web2c\updmap.cfg
%  INSTALLDIR\localtexmf\miktex\config\updmap.cfg
%\end{verbatim}
%
%Ensure the value for \texttt{pdftexDownloadBase14} is "true," and then follow the instructions found here:
%\httpAddr{//docs.miktex.org/manual/} to update your MikTeX installation.
%
%%------------------------------------------------------------------------
%\subsubsection*{Configuration Notes: Acrobat Distiller}
%
%We recommend to download and install the version of the ``CMW'' Adobe Acrobat Distiller job options file
%appropriate for your operating system and version of Acrobat from the following URL:
%
%\httpAddr{//www.cadmusmediaworks.com/index2.html}\\
%in the ``(Operating System)/Applications/Distiller Settings'' folder. The ``CMW'' job options file embeds
%all typefaces and does not downsample or subsample images when creating the PDF document.
%%------------------------------------------------------------------------
%\subsection{Copyright forms}
%
%You must include your signed Eurographics copyright release form
%when you submit your finished paper. We MUST have this form before
%your paper can be published in the proceedings.
%
%%-------------------------------------------------------------------------
%\subsection{Conclusions}
%
%Please direct any questions to the production editor in charge of
%these proceedings.
%
%%-------------------------------------------------------------------------
%
\bibliographystyle{abbrv}
\acknowledgments{
This work was supported by a grant from Department of Science and
Technology, India (SR/S3/EECE/0086/2012) and by the Robert Bosch
Centre for Cyber Physical Systems, Indian Institute of Science. Travel
was supported by IBM travel grant.
Volume rendered images
used in the paper were generated using Voreen (www.voreen.org).}
\bibliography{db}
%
%\bibliography{egbibsample}
%
%%-------------------------------------------------------------------------
%\newpage
%
%
%\begin{figure*}[tcb]
%  \centering
%  \mbox{} \hfill
%  % the following command controls the width of the embedded PS file
%  % (relative to the width of the current column)
%  \includegraphics[width=.3\linewidth]{sampleFig}
%  % replacing the above command with the one below will explicitly set
%  % the bounding box of the PS figure to the rectangle (xl,yl),(xh,yh).
%  % It will also prevent LaTeX from reading the PS file to determine
%  % the bounding box (i.e., it will speed up the compilation process)
%  % \includegraphics[width=.3\linewidth, bb=39 696 126 756]{sampleFig}
%  \hfill
%  \includegraphics[width=.3\linewidth]{sampleFig}
%  \hfill \mbox{}
%  \caption{\label{fig:ex3}%
%           For publications with color tables (i.e., publications not offering
%           color throughout the paper) please \textbf{observe}: 
%           for the printed version -- and ONLY for the printed
%           version -- color figures have to be placed in the last page.
%           \newline
%           For the electronic version, which will be converted to PDF before
%           making it available electronically, the color images should be
%           embedded within the document. Optionally, other multimedia
%           material may be attached to the electronic version. }
%\end{figure*}
%
\end{document}
