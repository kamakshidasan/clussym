%\documentclass[journal]{vgtc}                % final (journal style)
\documentclass[review,journal]{vgtc}         % review (journal style)
%\documentclass[widereview]{vgtc}             % wide-spaced review
%\documentclass[preprint,journal]{vgtc}       % preprint (journal style)
%\documentclass[electronic,journal]{vgtc}     % electronic version, journal

%% Uncomment one of the lines above depending on where your paper is
%% in the conference process. ``review'' and ``widereview'' are for review
%% submission, ``preprint'' is for pre-publication, and the final version
%% doesn't use a specific qualifier. Further, ``electronic'' includes
%% hyperreferences for more convenient online viewing.

%% Please use one of the ``review'' options in combination with the
%% assigned online id (see below) ONLY if your paper uses a double blind
%% review process. Some conferences, like IEEE Vis and InfoVis, have NOT
%% in the past.

%% Please note that the use of figures other than the optional teaser is not permitted on the first page
%% of the journal version.  Figures should begin on the second page and be
%% in CMYK or Grey scale format, otherwise, colour shifting may occur
%% during the printing process.  Papers submitted with figures other than the optional teaser on the
%% first page will be refused.

%% These three lines bring in essential packages: ``mathptmx'' for Type 1
%% typefaces, ``graphicx'' for inclusion of EPS figures. and ``times''
%% for proper handling of the times font family.

\usepackage{mathptmx}
\usepackage{graphicx}
\usepackage{times}
\usepackage{cite}
\usepackage{amsmath, amsfonts}
\usepackage{subfigure}
%\usepackage{algorithm2e}

%% We encourage the use of mathptmx for consistent usage of times font
%% throughout the proceedings. However, if you encounter conflicts
%% with other math-related packages, you may want to disable it.

%% This turns references into clickable hyperlinks.
\usepackage[bookmarks,backref=true,linkcolor=black]{hyperref} %,colorlinks
\hypersetup{
  pdfauthor = {},
  pdftitle = {},
  pdfsubject = {},
  pdfkeywords = {},
  colorlinks=true,
  linkcolor= black,
  citecolor= black,
  pageanchor=true,
  urlcolor = black,
  plainpages = false,
  linktocpage
}

%% If you are submitting a paper to a conference for review with a double
%% blind reviewing process, please replace the value ``0'' below with your
%% OnlineID. Otherwise, you may safely leave it at ``0''.
\onlineid{0}

%% declare the category of your paper, only shown in review mode
\vgtccategory{Technique}

%% allow for this line if you want the electronic option to work properly
\vgtcinsertpkg

%% In preprint mode you may define your own headline.
%\preprinttext{To appear in an IEEE VGTC sponsored conference.}

%% Paper title.
\title{Symmetry Detection in Scalar Fields by Clustering Contours}


%% This is how authors are specified in the journal style
\author{Dilip Mathew Thomas and Vijay Natarajan, \textit{Member, IEEE}}
\authorfooter{
% insert punctuation at end of each item
\item
 Dilip Mathew Thomas is with Department of Computer Science and Automation, Indian Institute of Science, Bangalore, India. E-mail: dilip@csa.iisc.ernet.in.
\item
 Vijay Natarajan is with Department of Computer Science and Automation, and Supercomputer Education Research Centre, Indian Institute of Science, Bangalore, India. E-mail: vijayn@csa.iisc.ernet.in.
%\item
% Martha Stewart is with Martha Stewart Enterprises at Microsoft
% Research, E-mail: martha.stewart@marthastewart.com.
}

%% indicate IEEE Member or Student Member in form indicated below
%\author{Roy G. Biv, Ed Grimley, \textit{Member, IEEE}, and Martha Stewart}
%\authorfooter{
%% insert punctuation at end of each item
%\item
% Roy G. Biv is with Starbucks Research. E-mail: roy.g.biv@aol.com.
%\item
% Ed Grimley is with Grimley Widgets, Inc.. E-mail: ed.grimley@aol.com.
%\item
% Martha Stewart is with Martha Stewart Enterprises at Microsoft
% Research. E-mail: martha.stewart@marthastewart.com.
%}

%other entries to be set up for journal
\shortauthortitle{Biv \MakeLowercase{\textit{et al.}}: }
%\shortauthortitle{Firstauthor \MakeLowercase{\textit{et al.}}: Paper Title}

%% Abstract section.
\abstract{} % end of abstract

%% Keywords that describe your work. Will show as 'Index Terms' in journal
%% please capitalize first letter and insert punctuation after last keyword
\keywords{}

%% ACM Computing Classification System (CCS). 
%% See <http://www.acm.org/class/1998/> for details.
%% The ``\CCScat'' command takes four arguments.

%\CCScatlist{ % not used in journal version
%	\CCScat{Computer Graphics}{I.3.6}{Computer Graphics}{Methodology and Techniques}
%}
\teaser{
	\centering
	\subfigure[]
	{
		\includegraphics[scale=.33]{figures/1897-full.png}
	}
	\subfigure[]
	{
		\scalebox{.33}{\input{xfig/desspace.pdf_t}}
	}
	\subfigure[]
	{
		\includegraphics[scale=.33]{figures/1897-sym.png}
	}
	\caption{\label{teaser}Clustering based analysis detects symmetry at different scales in the electron microscopy 
	data of AMP-activated kinase (EMDB-1897). (a)~Volume rendering of the data shows different symmetric patterns
	that exhibit 3-fold rotational symmetry. (b)~Contours 
	are represented as points in a high dimensional descriptor space (shown in 2D for illustration) using
	a suitable shape descriptor. Contours belonging to symmetric regions are also symmetric and have similar descriptors.
	Hence, symmetric contours form a cluster in the descriptor space and can be easily identified.  
	Three such clusters detected are shown in orange, blue, and pink. (c)~Symmetric regions in the volume 
	corresponding to the symmetric contours are extracted and three different symmetric regions are highlighted 
	in orange, blue, and pink corresponding to the three clusters.}
}

%% Uncomment below to include a teaser figure.
%%  \teaser{
%%  \centering
%%  \includegraphics[width=16cm]{CypressView}
%%  \caption{In the Clouds: Vancouver from Cypress Mountain.}
%%  }

%% Uncomment below to disable the manuscript note
%\renewcommand{\manuscriptnotetxt}{}

%% Copyright space is enabled by default as required by guidelines.
%% It is disabled by the 'review' option or via the following command:
% \nocopyrightspace

%%%%%%%%%%%%%%%%%%%%%%%%%%%%%%%%%%%%%%%%%%%%%%%%%%%%%%%%%%%%%%%%
%%%%%%%%%%%%%%%%%%%%%% START OF THE PAPER %%%%%%%%%%%%%%%%%%%%%%
%%%%%%%%%%%%%%%%%%%%%%%%%%%%%%%%%%%%%%%%%%%%%%%%%%%%%%%%%%%%%%%%%

\begin{document}

%% The ``\maketitle'' command must be the first command after the
%% ``\begin{document}'' command. It prepares and prints the title block.

%% the only exception to this rule is the \firstsection command
%-------------------------------------------------------------------------
\firstsection{Introduction}
\maketitle
Many scientific experiments and simulations generate scalar field data and 
they often contain symmetric or repeating patterns. In many disciplines, 
symmetry plays an important role in studying the underlying scientific phenomena. 
For example, in crystallography, symmetry information is used to determine 
the structure of a crystal~\cite{som07}. In product design, symmetry is important 
to ensure functional efficiency and optimal manufacturing cost~\cite{booth02}. Symmetry is a 
useful cue in biology for determining growth and development of organs~\cite{stev06}. Since 
the study of symmetric features is of great interest in scientific data 
analysis, the problem of detecting symmetry in scalar fields has received 
considerable attention among researchers in the recent past~\cite{ThomN11,HongS08,kerbWKS11,ThomN13,MasoodTN13}.

Automatic detection of symmetry in scalar fields is a challenging problem and 
the quest for a widely applicable, efficient, and robust method for symmetry 
detection is ongoing. Though symmetry detection in scalar
fields is a relatively new area of research, the problem of detecting symmetry
in shapes has been well studied in the geometry processing community.
These studies have established that clustering based analyses result in superior
performance in robust identification of symmetry. Some of the methods
that use clustering for detecting symmetry in shapes have been extended to
scalar fields. These methods operate by determining symmetry
transformations through aggregation of local symmetry of sample points of the
domain. Scalar field datasets are typically represented using scalar values 
assigned to a discrete set of sample points that represent the domain under 
consideration. However, the domain and the scalar values are assumed to be continuous by
interpolating the values at the sample points. Therefore, the sample points in a scalar 
field capture the lowest level of information. In practice, scientists are more interested
in higher level features, extracted through methods like segmentation and isosurface extraction,
for studying the underlying physical phenomena. Hence, symmetry detection methods based on 
local information available at the sample points face considerable difficulty in representing 
and extracting meaningful symmetric regions. Moreover, these methods are computationally 
expensive since the number of sample points in scalar field datasets is typically orders of magnitude
higher than that in geometric shape datasets.

Though it is clear from methods proposed in the geometry processing community
that a clustering based analysis offers significant advantages for symmetry detection,
we believe that unlike shapes, low-level information available at the sample points of the 
domain is not suited for detecting symmetry in scalar fields. In this work, we propose 
a novel symmetry detection method based on the idea of clustering contours. Isocontours 
are extensively used in studying scalar field datasets and contours capture information 
about a scalar field at a macroscale as opposed to sample points of the domain. Therefore, they
are more suitable for a clustering based analysis. It is easy to see that contours
belonging to regions with similar scalar field distribution are symmetric. Using an appropriate
shape descriptor, our method maps contours as points in the space of descriptors 
such that the distance between points in the descriptor space captures similarity 
relationship between the contours. As a result, points in the descriptor space 
corresponding to symmetric contours lie in close 
proximity to each other and form clusters in the descriptor space. The region of the domain
corresponding to each such contour can be extracted and these regions
are reported as symmetric as illustrated in Figure~\ref{teaser}. Note that the choice
of shape descriptor is not fixed and depending on noise characteristics and the similarity 
notion relevant to the application of interest, an appropriate descriptor may be used.

The main contributions of this paper are the following:
\begin{itemize}
\item A formulation of the problem of symmetry detection in scalar fields
as a clustering problem in a shape descriptor space. This model
provides a lot of flexibility in similarity analysis of scalar fields
as well as handling noise since it allows the shape representation and the 
descriptor space to be varied.
\item A novel representation of contours as points in a contour descriptor space.
Similarity between contours is naturally defined as the distance between points 
in this space. This is a generic representation that is of independent interest 
and is well suited for applications related to similarity analysis on scalar fields.
{\color{blue}I plan to show this using something along the lines of feature creation detection
application for time varying data.}
\item A robust algorithm to detect symmetric regions at multiple scales. Though geometry based 
symmetry detection methods are typically computationally costly, we design an efficient algorithm 
that employs elegant optimisations by incorporating topological information 
about the contours using the contour tree.
\item Applications to ...{\color{blue}hopefully, I will have interesting results
on asymmetry analysis of contours / feature creation detection in time varying data /
some application related to similarity analysis with certain specific choices of descriptors.}
\end{itemize}
Symmetry information in scalar fields
has been used for transfer function design, exploration of isosurfaces, selection of cross-section
planes and view directions, linked selection and editing, query driven exploration
and visualization of features through dual rendering~\cite{ThomN11,HongS08,ThomN13,MasoodTN13}.
We believe that as better techniques for robust detection of symmetry in scalar fields are developed
many more applications of symmetry detection will emerge.
\section{Related Work}
Existing symmetry detection methods in scalar fields can be broadly classified into two
categories, namely, geometry based methods and topology based methods. We briefly review
these methods in this section.
\subsection{Geometry based approaches}
Several methods have been proposed in the literature for detecting
symmetry in shapes and these methods are described in a survey paper by
Mitra~et~al.~\cite{MitPWC2012}. Some of these methods~\cite{Mitra06,KazhdanCDFR03,BokBWSS09} have 
been applied to scalar fields~\cite{MasoodTN13,HongS08,kerbWKS11}, however they struggle to address 
the challenges in extending geometric methods to scalar fields.
Scalar field datasets are significantly larger in size and hence symmetry detection is computationally costly.
Geometric methods typically consider geometric information derived from a small region around each sample
point of the domain for symmetry detection. Directly applying this approach to scalar field makes it difficult 
to capture important features in a scalar field and hence these methods perform poorly in extracting higher level
features of interest and handling noise in the data.
Moreover, the scalar values are assumed to be continuous over the domain by interpolating the values at 
the discrete set of sample points. Therefore, inspecting only local information at the sample points
introduces additional challeges due to discretization errors since the symmetric counterpart for a 
given point may be an interpolated point.

Hong and Shen~\cite{HongS08} propose a method to detect global reflective symmetry by identifying planes
of reflection that minimize the difference between the scalar value at a point and its
reflection. This method is computationally inefficient and cannot be easily extended to detect 
other types of symmetry. Kerber~et~al.~\cite{kerbWKS11} build a graph network of crease line features in a dataset
and compute geometric transformations that match subgraphs within the network to detect symmetry. 
Since only a small subset of features in scalar field datasets contain crease lines, 
this method is not very useful in practice. Masood~et~al.~\cite{MasoodTN13} detect
symmetry by identifying symmetry transformations as clusters in the space of all transformations.
The clusters are generated by aggregating local symmetry transformations of pairs of points in the 
domain. This method relies on local signatures of the sample points for determining 
transformations and as a result several parameters need to be tweaked at various stages of 
the symmetry detection pipeline to limit the adverse effects of variations in the local signatures 
and interpolation errors.  Moreover, the transformation space is often cluttered with a lot of 
spurious transformations and introduces artifacts in the extraction of symmetric regions from the point set. 
The above methods compute geometric transformations between candidate pairs for identifying
symmetry and are computationally costly. Moreover, these methods being purely driven by local geometric 
information, do not incorporate any notion of importance of a feature and consider every candidate pair 
to be equally important in their analysis. Our method, on the other hand, uses topological
information derived from the contour tree to infer importance of a feature and this allows the
design of a feature-aware algorithm for symmetry detection. 

Bruckner~et~al. propose 
an information theoretic approach for isosurface similarity detection~\cite{BrucknerM10,haidacher11}.
This method uses mutual information of distance transforms to quantify the information common
to two isosurfaces and build a similarity map between all pairs of isosurfaces. Clusters with high
mutual information is used to detect similar isosurfaces within and across datasets. Although
closely related to our method, by virtue of using distance transforms as the descriptor, 
this method identifies similar family of isosurfaces within a range of isovalues that 
form an onion-peel like layered arrangement whereas by identifying
similarity between contours and not isosurfaces, our method is more suited for identifying partial 
similarities and symmetric regions. While this method is limited to analysing similarity between
datasets only in pairs, the descriptor space can be used to analyse multiple datasets.
Similarity between different scalar fields has also been studied by measuring the extent of overlap
between contours~\cite{SchneiderWCHS08,schn13}. Changes in orientation affects the distance transform
descriptor and contour overlap and therefore the similarity estimation whereas our method is not tied 
to a particular choice of descriptor. Thus based on the requirements of the application under 
consideration, similarity estimation can be designed to be sensitive or insensitive to orientation. 
\subsection{Topology based methods}
Thomas and Natarajan propose topology based methods for symmetry detection and these methods 
are computationally efficient because they operate on graph representations of the scalar field
like the contour tree~\cite{ThomN11} and the extremum graph~\cite{ThomN13}. The contour tree based method 
assumes that the subtrees of the contour tree corresponding to symmetric regions are structurally
similar and detect symmetry by evaluating structural similarity between the subtrees using a 
similarity score. This method can detect symmetry at multiple scales but cannot handle
noise in scalar fields that destroy the repeating structure of the subtrees. The extremum 
graph based method selects a set of extrema called seed set and 
estimates distances robustly through a graph traversal procedure. A carefully chosen distance 
threshold is used to disconnect the graph and classify the seeds into different groups called 
super-seeds. A region growing procedure is then used to identify the symmetric region 
corresponding to each super-seed. This procedure makes a strong assumption that the symmetric regions
can be identified purely from the proximity relationship between the seeds. Hence,
it relies heavily on a meaningful selection of seed set which involves significant effort 
and understanding about the symmetry of the domain and also requires several thresholds
to be set.

The above two methods being topological in nature do not ensure that the regions detected by
them are indeed geometrically symmetric while our method being geometric in nature, 
ensures that the regions detected are symmetric. These methods also rely on pairwise
comparison between candidate regions to classify them into symmetric groups. Determining
the similarity threshold is a challenge when using datasets with varying characteristics.
Clustering based analysis avoids the need for pairwise comparisons and instead the symmetric regions
are directly obtained as clusters in the descriptor space. 
The contour tree provides metadata information about the contours
and allows integration of topological information in geometric processing and results in smarter
processing of the contours. Thus, by utilizing geometric information of contours and the power of contour tree 
as a topological abstraction that represents the entire family of contours in a scalar field, our method offers 
significant advantage over existing symmetry detection methods. Similarity between scalar fields have been
studied for shape matching applications using graph matching methods on discrete approximations of the contour 
tree~\cite{ZhangBKDNB06,HilagaSKK01}.

\section{Definitions}
Consider a \emph{scalar field}, $f : \mathbb{M}  \rightarrow \mathbb{R}$, defined on a 
simply connected domain $\mathbb{M}$. The preimage of $f$ for a given value $u \in \mathbb{R}$, 
$f^{-1}(u)$, is called a \emph{level set} of $f$. A level set may have multiple components
and each component is called a \emph{contour}. Consider a sweep of the level sets in the order
of increasing function values. A contour is created at a \emph{minimum}, may merge
with another contour or split into different contours at a \emph{saddle}, and is destroyed
at a \emph{maximum}. The \emph{contour tree} is a topological 
data structure that captures these changes in the connectivity of the level sets. 
Let $R$ be an equivalence relation defined on points in $\mathbb{M}$: $xRy$ for $x,y 
\in \mathbb{M}$  if $x,y$ belong to the same contour. The contour tree is the quotient space 
induced by this relation. 

Subdomains $\mathbb{M}_1, \mathbb{M}_2 \subseteq \mathbb{M}$ are said to be \emph{symmetric} 
if there is a transformation $T$ such that ${\mathbb{M}_2=T(\mathbb{M}_1)}$ and 
${f(x)=f(T(x))}$ for all $x \in \mathbb{M}_1$. If $c_1$ and $c_2$ are contours of the same level
set that belong to $\mathbb{M}_1$ and $\mathbb{M}_2$ respectively, it is easy to see that if 
$\mathbb{M}_1$ and $\mathbb{M}_2$ are symmetric then $c_1$ and $c_2$ are also symmetric. 
We make use of this property and detect symmetric subdomains by identifying symmetry of the 
contours belonging to the subdomains. The above definition of symmetry requires computation of the symmetry 
transformation $T$ which is a costly operation. Let $\mathbb{C}$ be the set of all contours. 
Consider a function $g : \mathbb{C} \rightarrow \mathbb{R}^n$ such that $g(c) = g(T(c))$
where $T$ is any transformation. In other words, $g$ is a function that maps each
contour to a point in a high dimensional space, called a \emph{descriptor}, 
such that a contour and its symmetric copies are mapped to the same point. 
We call this high dimensional space
\emph{descriptor space}. Thus, if $c_1$ and $c_2$ are perfectly symmetric,
the distance between them in the descriptor space is zero, i.e., 
$\lVert g(c_1)-g(c_2) \rVert = 0$, where $\lVert \cdot \rVert$ is a norm in
the descriptor space. Shape descriptors have been extensively used in the
geometry processing community for shape matching and retrieval and there is a
vast collection of research papers in this 
area~\cite{lian2013,van2011,tangelder2008survey,qin2008content}. Note that
different norms may have to be used for different descriptors. 

In practice, scalar field datasets do not exhibit perfect symmetry and therefore
it is important to detect symmetry in an approximate sense. Ideally, deviation
from perfect symmetry should be measured in the space of shapes but in practice,
it is more convenient to measure deviations in the descriptor space. If contours $c_1$ and 
$c_2$ deviate from perfect symmetry, then $c_1$ and $c_2$ will not be mapped
to the same point in the descriptor space and the distance between
them in the descriptor space, $\lVert g(c_1)-g(c_2) \rVert$ will be indicative
of the deviation from perfect symmetry. We say that $c_1$ and $c_2$ are
\emph{$\epsilon$-symmetric} if the distance between $c_1$ and $c_2$ in the descriptor 
space is less than $\epsilon$, i.e., $\lVert g(c_1)-g(c_2) \rVert < \epsilon$.
A shape descriptor may map contours $c_1$ and $c_2$ to the same point
even when they have totally different shapes. A good shape descriptor should 
discriminate

\section{Symmetry Detection via Contour Clustering}
Methods based on clustering~\cite{} have shown superior performance in detecting
symmetry in shapes. However, directly extending these methods to scalar fields
is not easy. In this section, we describe a novel symmetry detection method
based on clustering contours of the scalar field.
\subsection{Overview}
The main steps of our symmetry detection pipeline is shown in Fig.~\ref{}.
Given a scalar field as input, we generate a set of contours. For each contour
thus generated, a descriptor is computed. The descriptor for each contour
can be considered to be a point in a high dimensional space called the
descriptor space. The descriptor space is a transformation invariant space,
i.e., it reverses the effect of geometric transformation on contours and maps
symmetric contours to the same point in the descriptor space in the case of perfect
symmetry. Imperfections in symmetry results in imperfections in the mapping.
Since contours with similar shape have similar descriptors,
the points in the descriptor space corresponding to approximately symmetric 
contours will lie in close proximity to each other. Thus symmetric contours
can be detected by identifying clusters in the descriptor space. The volumetric
region enclosed by the contours in a cluster is then reported as
symmetric regions.
\subsection{Contour Generation}
Symmetric regions are detected by our method by identifying symmetry of the contours 
belonging to these regions. Hence, each region should be represented 
by a contour belonging to it. The obvious method of generating contours from isovalues
sampled uniformly from the range of the function values of the scalar field has several
drawbacks. A coarse uniform sampling may result in no contours getting generated from
a region and thus fail to detect it as a symmetric region while a fine sampling will lead
to multiple contours getting generated from the same region and hence result in redundant
computations. Ideally, each symmetric region should require only a single representative
contour belonging to the region to be processed.
The contour tree is a powerful tool that captures information about contours and we leverage
information obtained from the contour tree for optimal generation of contours. Each arc of
the contour tree represents a family of contours that are topologically equivalent. These
contours are nested one inside the other and each contour is typically a scaled version
of the contour nested inside it. Hence, to capture the geometry of the family
of contours corresponding to an arc of the contour tree, it is necessary to select only one
contour from each arc of the contour tree.

Selecting a representative contour from each arc of the contour tree ensures that no regions
are missed in the subsequent symmetry analysis. However, for noisy scalar fields, a large
number of arcs of the contour tree may correspond to noise. Selecting a contour from each
arc of the contour tree will result in significant amount of computational time spent in
processing these noisy contours and will lead to poor performance. To overcome this problem,
we generate a contour from from an arc of the contour tree only if the volume of the largest
contour associated with the arc is above a user defined threshold. This volume approximated
as the number of vertices of the domain enclosed by the contour. Since each arc of the contour
tree can be associated with the set of vertices of the domain that comprise the subvolume 
corresponding to the arc, this estimation of the volume can be done efficiently.
In the absence of the metadata information provided by the contour tree all contours would
have had to be treated as equally important and it would neither have been possible to ensure
that that each region had a representative contour nor suppress generation of
multiple contours from the same region or contours from noisy regions.
\subsection{Contour Representation in Descriptor Space}
Designing shape descriptors for shape matching is a well studied area in the 
geometry processing community. Shape matching is typically performed for pairs
of shapes by comparing the corresponding descriptors. As opposed to an approach based
on pairwise comparisons, a clustering based approach is more suited for symmetry
detection since it requires identification of a collection of similar regions.
In our method, each region is represented by a contour belonging to the region and symmetric
regions are detected by identifying symmetry of the representative contours. The descriptor
space is a space in which contours are represented as points in a transformation invariant
manner and hence symmetric contours are mapped to points that lie in close proximity to
each other. As a result, symmetric contours can be easily identified as clusters in the descriptor
space.

One of the major advantages of our method is that it can be used as a generic framework
for detecting similar regions in a scalar field. The notion of similarity is subjective
and varies from application to application and our method offers flexibility in choosing
the right descriptor for an application. Thus, if an application is interested in identifying
similarity only with respect to rotation, a rotation invariant descriptor may be used.
The only prerequisite on the descriptor is that it should be discriminative, i.e., contours
which are similar should be mapped to points that are nearby in the descriptor space while
contours that differ from each other should be mapped to far away points.
\subsection{Contour Clustering}
As described earlier, a set of contours can be mapped to points in a high
dimensional descriptor space and clusters in this space can be used to identify
symmetric contours. Strictly speaking, such a mapping of contours to points in the
descriptor space is not a prerequisite for clustering. A symmetry correspondence
graph can be constructed from a set of contours by representing each contour
as a node in the graph and inserting an edge between two nodes if the corresponding
contours are symmetric. It is easy to see that contours that are symmetric form
a clique under this representation and these cliques can be identified to detect
symmetric contours. Thus, given a function that assigns a similarity correspondence
between a pair of contours, the contours can be classified into clusters independent
of the actual definition of similarity which may be application dependent. For symmetry
detection, distance between points in the descriptor space may be used to assign
correspondence between pairs of contours. 

Given a set of contours identified to be symmetric, the arc in the contour tree 
corresponding to each of the contour can be determined and the region of the domain
enclosed by the largest contour of the arc can be extracted as the symmetric region.
Although this works well in practice, it is possible that an application needs to
consider the symmetry relationship of all the contours in a region to be incorporated 
in the symmetry detection algorithm. This presents a challenge in directly using 
clusters in the descriptor space for detecting symmetric region. Note  that the 
cluster in the descriptor space corresponding to a set of symmetric contours may 
not contain points corresponding to contours nested within the symmetric contours. 
This is because the descriptor space is not continuous with respect to the evolution 
of the shape of a contour during a level set sweep. For example, the contours shown in
blue in Fig.~\ref{} merge during the level set sweep and the shape of the contours 
after the merge is shown in green. Note that the shape of the contour after the merge 
is considerably different from the shape of the individual contours before the merge. 
As a result, the distance between points in the descriptor space corresponding to 
isosurface components just before and after a merge can be significantly high.

\begin{figure}[h]
\scalebox{.2}{\input{xfig/bip.pdf_t}}
\end{figure}
To address this issue, we incorporate the correspondence between the children contours 
(contours before merging) while calculating the correspondence between a given pair
of parent contours (contours after merging). Let $p$ and $q$ be two parent contours
with children contours ${c_1}^p,\dots,{c_n}^p$ and ${c_1}^q,\dots,{c_m}^q$ respectively. 
Assume that the correspondence between any two
children contours ${c_i}^p$ and ${c_j}^q$ is known - for contours with children,
the procedure below can be applied bottom up to determine their correspondence
while for contours that do not have children, the correspondence can be directly 
determined from the descriptor space. To calculate the correspondence 
between $p$ and $q$, first the contribution from the children contours 
is determined. We construct a complete bipartite graph where nodes in the first
and second partitions are ${c_1}^p,\dots,{c_n}^p$ and ${c_1}^q,\dots,{c_m}^q$ 
respectively. An edge between ${c_i}^p$ and ${c_j}^q$ is weighted
with the correspondence between ${c_i}^p$ and ${c_j}^q$. The maximum weight
matching can then be computed and gives the correspondence between the children
contours.
%All text with the exception of the abstract must be in a two-column format.
%The total allowable width of the text area -- including header and footer
%lines -- is 161\,mm (6.34 inch) wide by 231\,mm (9.10 inch) high.
%
%Columns are to be 76\,mm (3.0 inch) wide, with a 8\,mm (0.315 inch) space 
%between them.
%
%The space between the header line and the first line of the text body and
%between the last line of the text body and the footer line is 5\,mm
%(0.196 inch) each.
%
%%-------------------------------------------------------------------------
%\subsection{Type-style and fonts}
%
%Wherever Times is specified, Times Roman may also be used. If
%neither is available on your word processor, please use the font
%closest in appearance to Times that you have access to. Only
%Type-1 fonts will be accepted.
%
%MAIN TITLE. The title should be in Times 17-point, boldface type and
%centered. Capitalize the first letter of nouns, pronouns, verbs, adjectives,
%and adverbs; do not capitalize articles, coordinate conjunctions, or
%prepositions (unless the title begins with such a word). Leave two blank
%lines after the title.
%
%AUTHOR NAME(s) and AFFILIATION(s) are to be centered beneath the title and
%printed in Times 9-point, non-boldface type. This information is to be
%followed by two blank lines.
%
%The ABSTRACT ist to be in a one-column format. The MAIN TEXT is to be in a
%two-column format.
%
%MAIN TEXT. Type main text in 9-point Times, single-spaced. Do \emph{not} use
%double-spacing. All paragraphs should be indented 1 em (the length of the
%dash in the actual font). Make sure your text is fully justified -- that is,
%flush left and flush right. Please do not place any additional blank lines
%between paragraphs. Figure and table captions should be 9-point Times
%boldface type as in Figure~\ref{fig:firstExample}.
%
%\noindent Long captions should be set as in Figure~\ref{fig:ex1} or
%Fig.~\ref{fig:ex3}.
%
%\begin{figure}[htb]
%   % an empty figure just consisting of the caption lines
%   \caption{\label{fig:ex1}
%     'Empty' figure only to serve as an example of long caption requiring 
%     more than one line. It is not typed centered but aligned on both sides.}
%\end{figure}
%
%\noindent
%Figures which need the full textwidth can be typeset as Fig.~\ref{fig:ex3}.
%
%\noindent Callouts should be 9-point Times, non-boldface type. Initially
%capitalize only the first word of section titles and first-, second-, and
%third-order headings.
%
%FIRST-ORDER HEADINGS. (For example, \textbf{1. Introduction}) should be Times
%9-point boldface, initially capitalized, flush left, with one blank line
%before, and one blank line after.
%
%SECOND-ORDER HEADINGS. (For example, \textbf{2.1. Language}) should be Times
%9-point boldface, initially capitalized, flush left, with one blank line
%before, and one after. If you require a third-order heading (we discourage
%it), use 9-point Times, boldface, initially capitalized, flush left, preceded
%by one blank line, followed by a period and your text on the same line.
%
%The headline \emph{(authors / title)} must be shortened if it uses the full 
%two column width of the main text.
%There must be enough space for the page numbers. Please use ``et al.'' if 
%there are more than three authors and specify a shortened version for your title.
%%-------------------------------------------------------------------------
%\subsection{Footnotes}
%
%Please do \emph{not} use footnotes at all!
%
%
%%-------------------------------------------------------------------------
%\subsection{References}
%
%List all bibliographical references in 9-point Times, single-spaced, at the
%end of your paper in alphabetical order. When referenced in the text, enclose
%the citation index in square brackets, for example~\cite{Lous90}. Where
%appropriate, include the name(s) of editors of referenced books.
%
%For your references please use the following algorithm:
%\begin{itemize} 
%\item \textbf{one} author: first 3 chars plus year -- 
%      e.g.\ \cite{Lous90}
%\item \textbf{two}, \textbf{three} or \textbf{four} authors: first char
%      of each family name plus year --  e.g.\ \cite{Fellner-Helmberg93} 
%      or \cite{Kobbelt97-USHDR} or \cite{Lafortune97-NARF}
%\item \textbf{more than 4} authors: first char of family name from 
%      first 3 authors followed by a '*' followed by the year -- 
%%      e.g.\ \cite{Buhmann:1998:DCQ} or \cite{FolDamFeiHug.etal93} 
%\end{itemize}
%
%For BibTeX users a style file \ \texttt{eg-alpha.bst} \ is available which
%uses the above algorithm.
%
%%-------------------------------------------------------------------------
%\subsection{Illustrations, graphs, and photographs}
%
%All graphics should be centered.
%
%%%%
%%%% Figure 1
%%%%
%\begin{figure}[htb]
%  \centering
%  % the following command controls the width of the embedded PS file
%  % (relative to the width of the current column)
%  \includegraphics[width=.8\linewidth]{sampleFig}
%  % replacing the above command with the one below will explicitly set
%  % the bounding box of the PS figure to the rectangle (xl,yl),(xh,yh).
%  % It will also prevent LaTeX from reading the PS file to determine
%  % the bounding box (i.e., it will speed up the compilation process)
%  % \includegraphics[width=.95\linewidth, bb=39 696 126 756]{sampleFig}
%  %
%  \parbox[t]{.9\columnwidth}{\relax
%           For all figures please keep in mind that you \textbf{must not}
%           use images with transparent background! 
%           }
%  %
%  \caption{\label{fig:firstExample}
%           Here is a sample figure.}
%\end{figure}
%
%If your paper includes images, it is very important that they are of
%sufficient resolution to be faithfully reproduced.
%
%To determine the optimum size (width and height) of an image, measure
%the image's size as it appears in your document (in millimeters), and
%then multiply those two values by 12. The resulting values are the
%optimum $x$ and $y$ resolution, in pixels, of the image. Image quality
%will suffer if these guidelines are not followed.
%
%Example 1: 
%%
%An image measures 50\,mm by 75\,mm when placed in a document. This
%image should have a resolution of no less than 600 pixels by 900
%pixels in order to be reproduced faithfully.
%
%Example 2: 
%%
%Capturing a screenshot of your entire $1024 \times 768$ pixel display
%monitor may be useful in illustrating a concept from your research. In
%order to be reproduced faithfully, that $1024 \times 768$ image should
%be no larger than 85 mm by 64 mm (approximately) when placed in your
%document.
%
%
%%-------------------------------------------------------------------------
%\subsection{Color}
%
%\textbf{Please observe:} as of 2003 publications in the proceedings of the
%Eurographics Conference can use color images throughout the paper. No
%separate color tables are necessary.
%
%However, workshop proceedings might have different agreements! 
%Fig.~\ref{fig:ex3} is an example for creating color plates.
%
%%------------------------------------------------------------------------
%\subsection{Embedding of Hyperlinks / Typesetting of URLs}
%
%Due to the use of the package \texttt{hyperref} the original behavior
%of the command $\backslash$\texttt{url} from the package \texttt{url}
%is not available. To circumvent this problem we either recommend to
%use the command $\backslash$\texttt{httpAddr} from the 
%included package \texttt{egweblnk} (see below) or to replace the
%command $\backslash$\texttt{url} by the command $\backslash$\texttt{webLink} 
%-- e.g. in cases where $\backslash$\texttt{url} has been used
%widely in BibTeX-References. In the latter case we suggest to run
%BibTeX as usual and then replace all occurences of $\backslash$\texttt{url}  by
%$\backslash$\texttt{webLink}
%
%\noindent
%The provided commands for hyperlinks, in a nutshell, are:
%
%\begin{description} \itemsep 1ex
%\item [\webLinkFont $\backslash$httpAddr \{URL without leading 'http:'\}]
%      \mbox{}\\
%      e.g. \  \httpAddr{//diglib.eg.org/EG/DL/WS}
%
%\item[\webLinkFont $\backslash$ftpAddr \{URL without leading 'ftp:'\}]
%      \mbox{}\\
%      e.g. \  \ftpAddr{//www.eg.org/EG/DL/ftpupload}
%
%\item[\webLinkFont $\backslash$URL \{url\}]
%      \mbox{}\\
%      e.g. \  \URL{http://www.eg.org/EG/DL/WS}
%
%\item[\webLinkFont $\backslash$MailTo \{Email addr\}]
%      \mbox{}\\
%      e.g. \  \MailTo{publishing@eg.org}
%
%\item[\webLinkFont $\backslash$MailToNA \{emailName\}\{@emailSiteAddress\}]
%      \mbox{}\\
%      e.g. \  \MailToNA{publishing}{@eg.org}
%
%\item[\webLinkFont $\backslash$webLink\{URL without hyperlink creation\}]
%      \mbox{}\\
%      e.g. \  \webLink{http://www.eg.org/some_arbitrary_long/but_useless/URL}
%
%\end{description}
%
%
%%------------------------------------------------------------------------
%\subsection{PDF Generation}
%
%Your final paper should be delivered as a PDF document with all typefaces
%embedded. \LaTeX{} users should use \texttt{dvips} and \texttt{ps2pdf} to
%create this PDF document. Adobe Acrobat Distiller may be used in place of
%\texttt{ps2pdf}.
%
%Adobe PDFWriter is \emph{not} acceptable for use. Documents created with
%PDFWriter will be returned to the author for revision. \texttt{pdftex} and
%\texttt{pdflatex} (and its variants) can be used only if the author can
%make certain that all typefaces are embedded and images are not downsampled
%or subsampled during the PDF creation process.
%
%Users with no access to these PDF creation tools should make available a
%PostScript file and we will make a PDF document from it.
%
%
%The PDF file \emph{must not} be change protected.
%
%%------------------------------------------------------------------------
%\subsubsection*{Configuration Notes: dvips / ps2pdf / etc.}
%
%\noindent
%\texttt{dvips} should be invoked with the \texttt{-Ppdf} and \texttt{-G0}
%flags in order to use Type 1 PostScript typefaces:
%
%\begin{verbatim}
%    dvips -t a4 -Ppdf -G0 -o my.ps my.dvi
%\end{verbatim}
%
%
%\noindent
%If you are using version 7.x of GhostScript, please use the following method of invoking \texttt{ps2pdf}, in
%order to embed all typefaces and ensure that images are not downsampled or subsampled in the PDF
%creation process:
%
%\begin{verbatim}
%  ps2pdf -dMaxSubsetPct=100 \
%         -dCompatibilityLevel=1.3 \
%         -dSubsetFonts=true \
%         -dEmbedAllFonts=true \
%         -dAutoFilterColorImages=false \
%         -dAutoFilterGrayImages=false \
%         -dColorImageFilter=/FlateEncode \
%         -dGrayImageFilter=/FlateEncode \
%         -dMonoImageFilter=/FlateEncode \
%         mypaper.ps mypaper.pdf
%\end{verbatim}
%
%
%If you are using version 8.x of GhostScript, please use this method in place of the example above:
%\begin{verbatim}
%  ps2pdf -dPDFSETTINGS=/prepress \
%         -dCompatibilityLevel=1.3 \
%         -dAutoFilterColorImages=false \
%         -dAutoFilterGrayImages=false \
%         -dColorImageFilter=/FlateEncode \
%         -dGrayImageFilter=/FlateEncode \
%         -dMonoImageFilter=/FlateEncode \
%         -dDownsampleColorImages=false \
%         -dDownsampleGrayImages=false \
%         mypaper.ps mypaper.pdf
%\end{verbatim}
%
%%------------------------------------------------------------------------
%\subsubsection*{Configuration Notes: pdftex / pdflatex / etc.}
%
%\noindent
%Configuration of these tools to embed all typefaces can be accomplished by editing the \texttt{updmap.cfg} file
%to enable inclusion of the standard (or base) 14 typefaces.
%
%Linux users can run the \texttt{updmap} script to do this:
%\begin{verbatim}
%updmap --setoption pdftexDownloadBase14 true
%\end{verbatim}
%
%Windows users should edit the \texttt{updmap.cfg} files found in their TeX installation directories (one or both
%of the following may be present):
%\begin{verbatim}
%  INSTALLDIR\texmf\web2c\updmap.cfg
%  INSTALLDIR\localtexmf\miktex\config\updmap.cfg
%\end{verbatim}
%
%Ensure the value for \texttt{pdftexDownloadBase14} is "true," and then follow the instructions found here:
%\httpAddr{//docs.miktex.org/manual/} to update your MikTeX installation.
%
%%------------------------------------------------------------------------
%\subsubsection*{Configuration Notes: Acrobat Distiller}
%
%We recommend to download and install the version of the ``CMW'' Adobe Acrobat Distiller job options file
%appropriate for your operating system and version of Acrobat from the following URL:
%
%\httpAddr{//www.cadmusmediaworks.com/index2.html}\\
%in the ``(Operating System)/Applications/Distiller Settings'' folder. The ``CMW'' job options file embeds
%all typefaces and does not downsample or subsample images when creating the PDF document.
%%------------------------------------------------------------------------
%\subsection{Copyright forms}
%
%You must include your signed Eurographics copyright release form
%when you submit your finished paper. We MUST have this form before
%your paper can be published in the proceedings.
%
%%-------------------------------------------------------------------------
%\subsection{Conclusions}
%
%Please direct any questions to the production editor in charge of
%these proceedings.
%
%%-------------------------------------------------------------------------
%
\bibliographystyle{abbrv}
\acknowledgments{
This work was supported by a grant from Department of Science and
Technology, India (SR/S3/EECE/0086/2012) and by the Robert Bosch
Centre for Cyber Physical Systems, Indian Institute of Science. Travel
was supported by IBM travel grant.
Volume rendered images
used in the paper were generated using Voreen (www.voreen.org).}
\bibliography{db}
%
%\bibliography{egbibsample}
%
%%-------------------------------------------------------------------------
%\newpage
%
%
%\begin{figure*}[tcb]
%  \centering
%  \mbox{} \hfill
%  % the following command controls the width of the embedded PS file
%  % (relative to the width of the current column)
%  \includegraphics[width=.3\linewidth]{sampleFig}
%  % replacing the above command with the one below will explicitly set
%  % the bounding box of the PS figure to the rectangle (xl,yl),(xh,yh).
%  % It will also prevent LaTeX from reading the PS file to determine
%  % the bounding box (i.e., it will speed up the compilation process)
%  % \includegraphics[width=.3\linewidth, bb=39 696 126 756]{sampleFig}
%  \hfill
%  \includegraphics[width=.3\linewidth]{sampleFig}
%  \hfill \mbox{}
%  \caption{\label{fig:ex3}%
%           For publications with color tables (i.e., publications not offering
%           color throughout the paper) please \textbf{observe}: 
%           for the printed version -- and ONLY for the printed
%           version -- color figures have to be placed in the last page.
%           \newline
%           For the electronic version, which will be converted to PDF before
%           making it available electronically, the color images should be
%           embedded within the document. Optionally, other multimedia
%           material may be attached to the electronic version. }
%\end{figure*}
%
\end{document}
