%\documentclass[journal]{vgtc}                % final (journal style)
\documentclass[review,journal]{vgtc}         % review (journal style)
%\documentclass[widereview]{vgtc}             % wide-spaced review
%\documentclass[preprint,journal]{vgtc}       % preprint (journal style)
%\documentclass[electronic,journal]{vgtc}     % electronic version, journal

%% Uncomment one of the lines above depending on where your paper is
%% in the conference process. ``review'' and ``widereview'' are for review
%% submission, ``preprint'' is for pre-publication, and the final version
%% doesn't use a specific qualifier. Further, ``electronic'' includes
%% hyperreferences for more convenient online viewing.

%% Please use one of the ``review'' options in combination with the
%% assigned online id (see below) ONLY if your paper uses a double blind
%% review process. Some conferences, like IEEE Vis and InfoVis, have NOT
%% in the past.

%% Please note that the use of figures other than the optional teaser is not permitted on the first page
%% of the journal version.  Figures should begin on the second page and be
%% in CMYK or Grey scale format, otherwise, colour shifting may occur
%% during the printing process.  Papers submitted with figures other than the optional teaser on the
%% first page will be refused.

%% These three lines bring in essential packages: ``mathptmx'' for Type 1
%% typefaces, ``graphicx'' for inclusion of EPS figures. and ``times''
%% for proper handling of the times font family.

\usepackage{mathptmx}
\usepackage{graphicx}
\usepackage{times}
\usepackage{cite}
\usepackage{amsmath, amsfonts}
\usepackage{subfigure}
%\usepackage{algorithm2e}

%% We encourage the use of mathptmx for consistent usage of times font
%% throughout the proceedings. However, if you encounter conflicts
%% with other math-related packages, you may want to disable it.

%% This turns references into clickable hyperlinks.
\usepackage[bookmarks,backref=true,linkcolor=black]{hyperref} %,colorlinks
\hypersetup{
  pdfauthor = {},
  pdftitle = {},
  pdfsubject = {},
  pdfkeywords = {},
  colorlinks=true,
  linkcolor= black,
  citecolor= black,
  pageanchor=true,
  urlcolor = black,
  plainpages = false,
  linktocpage
}

%% If you are submitting a paper to a conference for review with a double
%% blind reviewing process, please replace the value ``0'' below with your
%% OnlineID. Otherwise, you may safely leave it at ``0''.
\onlineid{0}

%% declare the category of your paper, only shown in review mode
\vgtccategory{Technique}

%% allow for this line if you want the electronic option to work properly
\vgtcinsertpkg

%% In preprint mode you may define your own headline.
%\preprinttext{To appear in an IEEE VGTC sponsored conference.}

%% Paper title.
\title{Detecting Symmetry in Scalar Fields Using\\Augmented Extremum Graphs}


%% This is how authors are specified in the journal style
\author{Dilip Mathew Thomas and Vijay Natarajan, \textit{Member, IEEE}}
\authorfooter{
% insert punctuation at end of each item
\item
 Dilip Mathew Thomas is with Department of Computer Science and Automation, Indian Institute of Science, Bangalore, India. E-mail: dilip@csa.iisc.ernet.in.
\item
 Vijay Natarajan is with Department of Computer Science and Automation, and Supercomputer Education Research Centre, Indian Institute of Science, Bangalore, India. E-mail: vijayn@csa.iisc.ernet.in.
%\item
% Martha Stewart is with Martha Stewart Enterprises at Microsoft
% Research, E-mail: martha.stewart@marthastewart.com.
}

%% indicate IEEE Member or Student Member in form indicated below
%\author{Roy G. Biv, Ed Grimley, \textit{Member, IEEE}, and Martha Stewart}
%\authorfooter{
%% insert punctuation at end of each item
%\item
% Roy G. Biv is with Starbucks Research. E-mail: roy.g.biv@aol.com.
%\item
% Ed Grimley is with Grimley Widgets, Inc.. E-mail: ed.grimley@aol.com.
%\item
% Martha Stewart is with Martha Stewart Enterprises at Microsoft
% Research. E-mail: martha.stewart@marthastewart.com.
%}

%other entries to be set up for journal
\shortauthortitle{Biv \MakeLowercase{\textit{et al.}}: Detecting Symmetry in Scalar Fields Using Augmented Extremum Graphs}
%\shortauthortitle{Firstauthor \MakeLowercase{\textit{et al.}}: Paper Title}

%% Abstract section.
\abstract{Visualizing symmetric patterns in the data often helps the domain scientists make important observations 
	and gain insights about the underlying experiment. Detecting symmetry in scalar fields is a nascent area of
	research and existing methods that detect symmetry are either not robust in the presence of noise or 
	computationally costly. We propose a data structure called the augmented extremum graph 
	and use it to design a novel symmetry detection method based on robust estimation of distances.
	The augmented extremum graph captures both topological and geometric information of the scalar field
	and enables robust and computationally efficient detection of symmetry.
	We apply the proposed method to detect symmetries in cryo-electron microscopy datasets and the
	experiments demonstrate that the algorithm is capable of detecting symmetry even in the presence of significant noise.
	We describe novel applications that use the detected symmetry to enhance visualization of scalar field data and 
	facilitate their exploration.
} % end of abstract

%% Keywords that describe your work. Will show as 'Index Terms' in journal
%% please capitalize first letter and insert punctuation after last keyword
\keywords{Scalar field visualization, extremum graph, Morse decomposition, symmetry detection, data exploration.}

%% ACM Computing Classification System (CCS). 
%% See <http://www.acm.org/class/1998/> for details.
%% The ``\CCScat'' command takes four arguments.

%\CCScatlist{ % not used in journal version
%	\CCScat{Computer Graphics}{I.3.6}{Computer Graphics}{Methodology and Techniques}
%}
\teaser{
	\centering
	\subfigure
	{
		\includegraphics[scale=.295]{figures/1654-full.png}
	}
	\subfigure
	{
		\includegraphics[scale=.295]{figures/1654-seeds.png}
	}
	\subfigure
	{
		\includegraphics[scale=.295]{figures/1654-sym.png}
	}
	\caption{\label{teaser}Robust scalar field symmetry identification algorithm detects symmetry even in the presence of significant noise
	in the electron microscopy data of the Rubisco RbcL8-RbcX2-8 complex (EMDB 1654). (left)~Volume rendering 
	shows symmetry and noise in the data. (center)~A set of seed cells is chosen as source vertices for traversing
	the augmented extremum graph of the data. During the traversal, the seed cells merge together to form four
	symmetric super-seeds. Seed cells that belong to a common super-seed are shown with the same color.
	(right)~The initial estimate of symmetry is expanded in a region growing stage to identify the symmetric regions.
	A symmetry-aware transfer function highlights the 4-fold rotational symmetry detected in the Rubisco complex.}
}

%% Uncomment below to include a teaser figure.
%%  \teaser{
%%  \centering
%%  \includegraphics[width=16cm]{CypressView}
%%  \caption{In the Clouds: Vancouver from Cypress Mountain.}
%%  }

%% Uncomment below to disable the manuscript note
%\renewcommand{\manuscriptnotetxt}{}

%% Copyright space is enabled by default as required by guidelines.
%% It is disabled by the 'review' option or via the following command:
% \nocopyrightspace

%%%%%%%%%%%%%%%%%%%%%%%%%%%%%%%%%%%%%%%%%%%%%%%%%%%%%%%%%%%%%%%%
%%%%%%%%%%%%%%%%%%%%%% START OF THE PAPER %%%%%%%%%%%%%%%%%%%%%%
%%%%%%%%%%%%%%%%%%%%%%%%%%%%%%%%%%%%%%%%%%%%%%%%%%%%%%%%%%%%%%%%%

\begin{document}

%% The ``\maketitle'' command must be the first command after the
%% ``\begin{document}'' command. It prepares and prints the title block.

%% the only exception to this rule is the \firstsection command
\firstsection{Introduction}

\maketitle


%-------------------------------------------------------------------------
Many scientific experiments and simulations generate scalar field data and 
they often contain symmetric or repeating patterns. In many disciplines, 
symmetry plays an important role in studying the underlying scientific phenomena. 
For example, in crystallography, symmetry information is used to determine 
the structure of a crystal. In product design, symmetry is important to ensure
functional efficiency and optimal manufacturing cost. Symmetry is a 
useful cue in bilogoy for determining growth and development of organs. Since 
the study of symmetric features is of great interest in scientific data 
analysis, the problem of detecting symmetry in scalar fields has received 
considerable attention among researchers in the recent past.

Automatic detection of symmetry in scalar fields is a challenging problem and 
the quest for a widely applicable, efficient, and robust method for symmetry 
detection has remained largely unanswered. Though symmetry detection in scalar
fields is a relatively new area of research, the problem of detecting symmetry
in shapes has been well studied in the geometry processing community.
These studies have established that clustering based analysis show superior
performance in robust identification of symmetry. Although some of the methods
that use clustering for detecting symmetry in shapes have been extended to
scalar fields also, these methods operate by determining symmetry
transformations through aggregation of local symmetry of sample points of the
domain. Since sample points in a scalar field capture information at a
microscale granularity, these methods face difficulties in extracting and
representing symmetric regions and are also computationally inefficient.

In this work, we propose a novel symmetry detection method based on the idea of
clustering contours. We believe that contours capture information about a
scalar field at a macroscale as opposed to points of the domain and hence they
are a more meaningful entity for performing a clustering based analysis for
symmetry detection. Existing methods that use clustering for detecting symmetry
in scalar fields are based on determining symmetry transformations which is 
computationally expensive. Our method, on the other hand, maps contours as
points in a descriptor space such that similar contours lie in close proximity
to each other.  Thus, the distance between points in the descriptor space 
captures similarity relationship between the contours and clusters in the 
descriptor space can be used to determine symmetric regions as illustrated in
Figure~\ref{teaser}. 

The main contributions of this paper are the following:
\begin{itemize}
\item Modelling the problem of symmetry detection in scalar fields
as a clustering problem in a descriptor space.
\end{itemize}
Symmetry information in scalar fields
has been used for transfer function design, exploration of isosurfaces, selection of cross-section
planes and view directions, linked selection and editing, query driven exploration
and visualization of features through dual rendering~\cite{}.
We believe that as better techniques for robust detection of symmetry in scalar fields emerge,
many more applications of symmetry detection will be explored in the future.


%Computational efficiency is achieved by using a topological data structure called extremum graphs,
%which provide a compact representation of the scalar field. An abstract representation of the extremum
%graph does not explicitly capture the geometry of the scalar field. 
%Hence, we augment the extremum graph with geometric information and this results in a geometry-aware
%symmetry detection method. 
%Scalar field datasets are typically
%large in size and often noisy. Even though a scalar field is represented
%using a discrete set of points, it is often assumed to be 
%continuous over the domain after interpolating the discrete set of scalar values.
%Hence, for a given point, its symmetric counterpart may be an interpolated point.
%Therefore, it is important to consider scalar values of the interpolated points
%unlike the case of geometric methods 
%which typically consider only the given discrete set of points for symmetry detection.
%However, it is still natural to consider extensions of geometric techniques to scalar fields.
%\noindent{\textbf{Advantages over contour tree.}} Our method has significant advantages over 
%the symmetry detection method proposed recently by Thomas and Natarajan, which is 
%based on detecting similar subtrees of the branch decomposition representation of 
%the contour tree~\cite{ThomN11}. Since branches corresponding to noise in the data can destroy 
%the similarity of the subtrees, their method requires the removal of such branches. For this purpose, 
%the branch decomposition is simplified by removing low persistence branches under the assumption that 
%only low amplitude noise exists in the data. Though our method also simplifies the extremum graph for 
%computational efficiency, in contrast to the assumption made by the contour tree based method, we do not necessarily
%require that the noise is removed through the simplification step. As described above, the presence of noise does not 
%affect the distance calculation and hence our method can handle noise of larger amplitude in the data.
%Moreover, our method incorporates geometric information for more effective symmetry detection and also uses a region 
%growing procedure to identify the largest symmetric 
%region. In comparison, the method by Thomas and Natarajan ignores geometric information and does not necessarily
%identify the largest symmetric region.
%In Fig.~\ref{extterrain}, depending on the noise in the data, the branches corresponding to the noisy hills
%can merge with the branches corresponding to the mountains in an arbitrary manner.
%The violet and the blue subtrees in Fig.~\ref{terrain-bd} 
%show the branch decomposition representation of the symmetric regions in Fig.~\ref{extterrain}.
%Note that the two subtrees are quite different and hence the subtree matching method cannot
%identify the symmetric regions.
%We illustrate this further with a real-world dataset in Section~\ref{resanddis}.
%In this section, we first present an overview of the proposed method by describing the
%different stages in the symmetry detection pipeline. Next, we describe the core of the pipeline, 
%namely traversal of the augmented extremum graph and formation of super-seeds. \subsection{Overview}
%of detecting symmetry in scalar fields and we believe that several more applications will be
%explored in the future, not only for visualization, but also for specific domains
%that can exploit symmetry information.

%All text with the exception of the abstract must be in a two-column format.
%The total allowable width of the text area -- including header and footer
%lines -- is 161\,mm (6.34 inch) wide by 231\,mm (9.10 inch) high.
%
%Columns are to be 76\,mm (3.0 inch) wide, with a 8\,mm (0.315 inch) space 
%between them.
%
%The space between the header line and the first line of the text body and
%between the last line of the text body and the footer line is 5\,mm
%(0.196 inch) each.
%
%%-------------------------------------------------------------------------
%\subsection{Type-style and fonts}
%
%Wherever Times is specified, Times Roman may also be used. If
%neither is available on your word processor, please use the font
%closest in appearance to Times that you have access to. Only
%Type-1 fonts will be accepted.
%
%MAIN TITLE. The title should be in Times 17-point, boldface type and
%centered. Capitalize the first letter of nouns, pronouns, verbs, adjectives,
%and adverbs; do not capitalize articles, coordinate conjunctions, or
%prepositions (unless the title begins with such a word). Leave two blank
%lines after the title.
%
%AUTHOR NAME(s) and AFFILIATION(s) are to be centered beneath the title and
%printed in Times 9-point, non-boldface type. This information is to be
%followed by two blank lines.
%
%The ABSTRACT ist to be in a one-column format. The MAIN TEXT is to be in a
%two-column format.
%
%MAIN TEXT. Type main text in 9-point Times, single-spaced. Do \emph{not} use
%double-spacing. All paragraphs should be indented 1 em (the length of the
%dash in the actual font). Make sure your text is fully justified -- that is,
%flush left and flush right. Please do not place any additional blank lines
%between paragraphs. Figure and table captions should be 9-point Times
%boldface type as in Figure~\ref{fig:firstExample}.
%
%\noindent Long captions should be set as in Figure~\ref{fig:ex1} or
%Fig.~\ref{fig:ex3}.
%
%\begin{figure}[htb]
%   % an empty figure just consisting of the caption lines
%   \caption{\label{fig:ex1}
%     'Empty' figure only to serve as an example of long caption requiring 
%     more than one line. It is not typed centered but aligned on both sides.}
%\end{figure}
%
%\noindent
%Figures which need the full textwidth can be typeset as Fig.~\ref{fig:ex3}.
%
%\noindent Callouts should be 9-point Times, non-boldface type. Initially
%capitalize only the first word of section titles and first-, second-, and
%third-order headings.
%
%FIRST-ORDER HEADINGS. (For example, \textbf{1. Introduction}) should be Times
%9-point boldface, initially capitalized, flush left, with one blank line
%before, and one blank line after.
%
%SECOND-ORDER HEADINGS. (For example, \textbf{2.1. Language}) should be Times
%9-point boldface, initially capitalized, flush left, with one blank line
%before, and one after. If you require a third-order heading (we discourage
%it), use 9-point Times, boldface, initially capitalized, flush left, preceded
%by one blank line, followed by a period and your text on the same line.
%
%The headline \emph{(authors / title)} must be shortened if it uses the full 
%two column width of the main text.
%There must be enough space for the page numbers. Please use ``et al.'' if 
%there are more than three authors and specify a shortened version for your title.
%%-------------------------------------------------------------------------
%\subsection{Footnotes}
%
%Please do \emph{not} use footnotes at all!
%
%
%%-------------------------------------------------------------------------
%\subsection{References}
%
%List all bibliographical references in 9-point Times, single-spaced, at the
%end of your paper in alphabetical order. When referenced in the text, enclose
%the citation index in square brackets, for example~\cite{Lous90}. Where
%appropriate, include the name(s) of editors of referenced books.
%
%For your references please use the following algorithm:
%\begin{itemize} 
%\item \textbf{one} author: first 3 chars plus year -- 
%      e.g.\ \cite{Lous90}
%\item \textbf{two}, \textbf{three} or \textbf{four} authors: first char
%      of each family name plus year --  e.g.\ \cite{Fellner-Helmberg93} 
%      or \cite{Kobbelt97-USHDR} or \cite{Lafortune97-NARF}
%\item \textbf{more than 4} authors: first char of family name from 
%      first 3 authors followed by a '*' followed by the year -- 
%%      e.g.\ \cite{Buhmann:1998:DCQ} or \cite{FolDamFeiHug.etal93} 
%\end{itemize}
%
%For BibTeX users a style file \ \texttt{eg-alpha.bst} \ is available which
%uses the above algorithm.
%
%%-------------------------------------------------------------------------
%\subsection{Illustrations, graphs, and photographs}
%
%All graphics should be centered.
%
%%%%
%%%% Figure 1
%%%%
%\begin{figure}[htb]
%  \centering
%  % the following command controls the width of the embedded PS file
%  % (relative to the width of the current column)
%  \includegraphics[width=.8\linewidth]{sampleFig}
%  % replacing the above command with the one below will explicitly set
%  % the bounding box of the PS figure to the rectangle (xl,yl),(xh,yh).
%  % It will also prevent LaTeX from reading the PS file to determine
%  % the bounding box (i.e., it will speed up the compilation process)
%  % \includegraphics[width=.95\linewidth, bb=39 696 126 756]{sampleFig}
%  %
%  \parbox[t]{.9\columnwidth}{\relax
%           For all figures please keep in mind that you \textbf{must not}
%           use images with transparent background! 
%           }
%  %
%  \caption{\label{fig:firstExample}
%           Here is a sample figure.}
%\end{figure}
%
%If your paper includes images, it is very important that they are of
%sufficient resolution to be faithfully reproduced.
%
%To determine the optimum size (width and height) of an image, measure
%the image's size as it appears in your document (in millimeters), and
%then multiply those two values by 12. The resulting values are the
%optimum $x$ and $y$ resolution, in pixels, of the image. Image quality
%will suffer if these guidelines are not followed.
%
%Example 1: 
%%
%An image measures 50\,mm by 75\,mm when placed in a document. This
%image should have a resolution of no less than 600 pixels by 900
%pixels in order to be reproduced faithfully.
%
%Example 2: 
%%
%Capturing a screenshot of your entire $1024 \times 768$ pixel display
%monitor may be useful in illustrating a concept from your research. In
%order to be reproduced faithfully, that $1024 \times 768$ image should
%be no larger than 85 mm by 64 mm (approximately) when placed in your
%document.
%
%
%%-------------------------------------------------------------------------
%\subsection{Color}
%
%\textbf{Please observe:} as of 2003 publications in the proceedings of the
%Eurographics Conference can use color images throughout the paper. No
%separate color tables are necessary.
%
%However, workshop proceedings might have different agreements! 
%Fig.~\ref{fig:ex3} is an example for creating color plates.
%
%%------------------------------------------------------------------------
%\subsection{Embedding of Hyperlinks / Typesetting of URLs}
%
%Due to the use of the package \texttt{hyperref} the original behavior
%of the command $\backslash$\texttt{url} from the package \texttt{url}
%is not available. To circumvent this problem we either recommend to
%use the command $\backslash$\texttt{httpAddr} from the 
%included package \texttt{egweblnk} (see below) or to replace the
%command $\backslash$\texttt{url} by the command $\backslash$\texttt{webLink} 
%-- e.g. in cases where $\backslash$\texttt{url} has been used
%widely in BibTeX-References. In the latter case we suggest to run
%BibTeX as usual and then replace all occurences of $\backslash$\texttt{url}  by
%$\backslash$\texttt{webLink}
%
%\noindent
%The provided commands for hyperlinks, in a nutshell, are:
%
%\begin{description} \itemsep 1ex
%\item [\webLinkFont $\backslash$httpAddr \{URL without leading 'http:'\}]
%      \mbox{}\\
%      e.g. \  \httpAddr{//diglib.eg.org/EG/DL/WS}
%
%\item[\webLinkFont $\backslash$ftpAddr \{URL without leading 'ftp:'\}]
%      \mbox{}\\
%      e.g. \  \ftpAddr{//www.eg.org/EG/DL/ftpupload}
%
%\item[\webLinkFont $\backslash$URL \{url\}]
%      \mbox{}\\
%      e.g. \  \URL{http://www.eg.org/EG/DL/WS}
%
%\item[\webLinkFont $\backslash$MailTo \{Email addr\}]
%      \mbox{}\\
%      e.g. \  \MailTo{publishing@eg.org}
%
%\item[\webLinkFont $\backslash$MailToNA \{emailName\}\{@emailSiteAddress\}]
%      \mbox{}\\
%      e.g. \  \MailToNA{publishing}{@eg.org}
%
%\item[\webLinkFont $\backslash$webLink\{URL without hyperlink creation\}]
%      \mbox{}\\
%      e.g. \  \webLink{http://www.eg.org/some_arbitrary_long/but_useless/URL}
%
%\end{description}
%
%
%%------------------------------------------------------------------------
%\subsection{PDF Generation}
%
%Your final paper should be delivered as a PDF document with all typefaces
%embedded. \LaTeX{} users should use \texttt{dvips} and \texttt{ps2pdf} to
%create this PDF document. Adobe Acrobat Distiller may be used in place of
%\texttt{ps2pdf}.
%
%Adobe PDFWriter is \emph{not} acceptable for use. Documents created with
%PDFWriter will be returned to the author for revision. \texttt{pdftex} and
%\texttt{pdflatex} (and its variants) can be used only if the author can
%make certain that all typefaces are embedded and images are not downsampled
%or subsampled during the PDF creation process.
%
%Users with no access to these PDF creation tools should make available a
%PostScript file and we will make a PDF document from it.
%
%
%The PDF file \emph{must not} be change protected.
%
%%------------------------------------------------------------------------
%\subsubsection*{Configuration Notes: dvips / ps2pdf / etc.}
%
%\noindent
%\texttt{dvips} should be invoked with the \texttt{-Ppdf} and \texttt{-G0}
%flags in order to use Type 1 PostScript typefaces:
%
%\begin{verbatim}
%    dvips -t a4 -Ppdf -G0 -o my.ps my.dvi
%\end{verbatim}
%
%
%\noindent
%If you are using version 7.x of GhostScript, please use the following method of invoking \texttt{ps2pdf}, in
%order to embed all typefaces and ensure that images are not downsampled or subsampled in the PDF
%creation process:
%
%\begin{verbatim}
%  ps2pdf -dMaxSubsetPct=100 \
%         -dCompatibilityLevel=1.3 \
%         -dSubsetFonts=true \
%         -dEmbedAllFonts=true \
%         -dAutoFilterColorImages=false \
%         -dAutoFilterGrayImages=false \
%         -dColorImageFilter=/FlateEncode \
%         -dGrayImageFilter=/FlateEncode \
%         -dMonoImageFilter=/FlateEncode \
%         mypaper.ps mypaper.pdf
%\end{verbatim}
%
%
%If you are using version 8.x of GhostScript, please use this method in place of the example above:
%\begin{verbatim}
%  ps2pdf -dPDFSETTINGS=/prepress \
%         -dCompatibilityLevel=1.3 \
%         -dAutoFilterColorImages=false \
%         -dAutoFilterGrayImages=false \
%         -dColorImageFilter=/FlateEncode \
%         -dGrayImageFilter=/FlateEncode \
%         -dMonoImageFilter=/FlateEncode \
%         -dDownsampleColorImages=false \
%         -dDownsampleGrayImages=false \
%         mypaper.ps mypaper.pdf
%\end{verbatim}
%
%%------------------------------------------------------------------------
%\subsubsection*{Configuration Notes: pdftex / pdflatex / etc.}
%
%\noindent
%Configuration of these tools to embed all typefaces can be accomplished by editing the \texttt{updmap.cfg} file
%to enable inclusion of the standard (or base) 14 typefaces.
%
%Linux users can run the \texttt{updmap} script to do this:
%\begin{verbatim}
%updmap --setoption pdftexDownloadBase14 true
%\end{verbatim}
%
%Windows users should edit the \texttt{updmap.cfg} files found in their TeX installation directories (one or both
%of the following may be present):
%\begin{verbatim}
%  INSTALLDIR\texmf\web2c\updmap.cfg
%  INSTALLDIR\localtexmf\miktex\config\updmap.cfg
%\end{verbatim}
%
%Ensure the value for \texttt{pdftexDownloadBase14} is "true," and then follow the instructions found here:
%\httpAddr{//docs.miktex.org/manual/} to update your MikTeX installation.
%
%%------------------------------------------------------------------------
%\subsubsection*{Configuration Notes: Acrobat Distiller}
%
%We recommend to download and install the version of the ``CMW'' Adobe Acrobat Distiller job options file
%appropriate for your operating system and version of Acrobat from the following URL:
%
%\httpAddr{//www.cadmusmediaworks.com/index2.html}\\
%in the ``(Operating System)/Applications/Distiller Settings'' folder. The ``CMW'' job options file embeds
%all typefaces and does not downsample or subsample images when creating the PDF document.
%%------------------------------------------------------------------------
%\subsection{Copyright forms}
%
%You must include your signed Eurographics copyright release form
%when you submit your finished paper. We MUST have this form before
%your paper can be published in the proceedings.
%
%%-------------------------------------------------------------------------
%\subsection{Conclusions}
%
%Please direct any questions to the production editor in charge of
%these proceedings.
%
%%-------------------------------------------------------------------------
%
%\bibliographystyle{eg-alpha}
\acknowledgments{
This work was supported by a grant from Department of Science and
Technology, India (SR/S3/EECE/0086/2012) and by the Robert Bosch
Centre for Cyber Physical Systems, Indian Institute of Science. Travel
was supported by IBM travel grant.
Volume rendered images
used in the paper were generated using Voreen (www.voreen.org).}
\bibliographystyle{abbrv}
\bibliography{db}
%
%\bibliography{egbibsample}
%
%%-------------------------------------------------------------------------
%\newpage
%
%
%\begin{figure*}[tcb]
%  \centering
%  \mbox{} \hfill
%  % the following command controls the width of the embedded PS file
%  % (relative to the width of the current column)
%  \includegraphics[width=.3\linewidth]{sampleFig}
%  % replacing the above command with the one below will explicitly set
%  % the bounding box of the PS figure to the rectangle (xl,yl),(xh,yh).
%  % It will also prevent LaTeX from reading the PS file to determine
%  % the bounding box (i.e., it will speed up the compilation process)
%  % \includegraphics[width=.3\linewidth, bb=39 696 126 756]{sampleFig}
%  \hfill
%  \includegraphics[width=.3\linewidth]{sampleFig}
%  \hfill \mbox{}
%  \caption{\label{fig:ex3}%
%           For publications with color tables (i.e., publications not offering
%           color throughout the paper) please \textbf{observe}: 
%           for the printed version -- and ONLY for the printed
%           version -- color figures have to be placed in the last page.
%           \newline
%           For the electronic version, which will be converted to PDF before
%           making it available electronically, the color images should be
%           embedded within the document. Optionally, other multimedia
%           material may be attached to the electronic version. }
%\end{figure*}
%
\end{document}
